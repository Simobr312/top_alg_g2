\documentclass[]{article}

% Packages
\usepackage[utf8]{inputenc} % For UTF-8 encoding
\usepackage[T1]{fontenc}    % For proper Italian accents
\usepackage[italian]{babel} % Italian language support
\usepackage{amsmath, amsfonts, amssymb} % Math packages
\usepackage{graphicx}       % For including images
\usepackage{hyperref}       % For hyperlinks
\usepackage{bookmark}       % For improved PDF bookmarks and rerunfilecheck warning
\usepackage{tikz}           % For diagrams
\usepackage{geometry}       % For page layout
\usepackage{fancyhdr}       % For custom headers/footers
\usepackage{enumitem}       % For custom lists
\usepackage{xcolor}         % For colored text
\usepackage{mathtools}      % For additional math tools
\usepackage{amsthm}         % For theorem environments
\usepackage{quiver}         % For commutative diagrams

% Load custom commands
\usepackage{commands} 
% Theorem styles

% Page layout
\geometry{a4paper, margin=1in}
\pagestyle{fancy}
\fancyhf{}
\fancyhead[L]{Note della parte di Topologia Algebrica del corso di Geometria 2 del 2024/25}
\fancyhead[R]{\thepage}

\begin{document}

\title{Appunti di Topologia Algebrica del corso di Geometria 2}
\author{Simone Riccio \\ \small{(dalle lezioni del Prof. Tamas Szamuely)}}
\date{\today}

\maketitle

\tableofcontents

\section{Gruppo Fondamentale}
\subsection{Omotopia}
    Una delle motivazioni che porta a definire il gruppo fondamentale è la necessità di distinguere
    due spazi topologici a meno di omeomorfismo.

    \begin{example} \nl
        Si consideri il disco 
        \[  
            D^n := \set{ x \in \R^{n} \mid \norm{x} \leq 1 }
        \]
        Al variare di $n$ naturale i $D^n$ non sono intuitivamente omeomorfi, tuttavia dimostrarlo
        usando solo la topologia generale è difficile.

        È semplice mostrare che $\Disc{1} \not\cong D^n$ per $n \geq 2$, usando l'insieme delle componenti 
        connesse. Infatti, per ogni $x \in D^n$ lo spazio topologico $D^n \setminus \set{x}$ è connesso 
        per ogni $n \geq 2$, mentre $\Disc{1} \setminus \set{x}$, essendo il segmento $[-1,1]$ senza un punto, ha due componenti connesse.
    
        Tale argomentazione non funziona già per provare a distinguere $\Disc{2}$ dai $D^n$ con $n \geq 3$.
        Introduciamo quindi il gruppo fondamentale, che permetterà in futuro di distinguerli tutti.
    \end{example}

    \begin{definition}[Omotopia] \nl
        Date due funzioni continue $f,g: X \to Y$ tra spazi topologici, si dice che $f$ e $g$ sono \textbf{omotope} se esiste una funzione
        \[
            H: I \times X \to Y
        \]
        \textbf{continua} e tale che:
        \begin{itemize}
            \item $H(0,x) = f(x)$ per ogni $x \in X$;
            \item $H(1,x) = g(x)$ per ogni $x \in X$;
            \item $H(s,y) = H(s,x)$ per ogni $s \in I$ e per ogni $x,y \in X$ tali che $f(x) = f(y)$.
        \end{itemize}
        Si dice che $H$ è un'\textbf{omotopia} tra $f$ e $g$ e si scrive
        \[
            f \sim g.
        \]
        Inoltre si pu\`o vedere un'omotopia come una famiglia di funzioni contiune:
        \[
            \set{f_s: X \to Y}_{s \in I}
            \quad \text{con } f_s(x) = H(s,x).
        \]
        Che rappresentano una deformazione continua di $f$ in $g$.
    \end{definition}

    \begin{definition}[Omotopia di cammini a estremi fissi] \nl
        Due cammini $\gamma_0, \gamma_1: I \to X$ si dicono \textbf{omotopi (a estremi fissi)} se esiste una funzione
        \[
            H: I \times I \to X
        \]
        \textbf{continua} e tale che:
        \begin{itemize}
            \item $H(0,t) = \gamma_0(t)$ per ogni $t \in I$;
            \item $H(1,t) = \gamma_1(t)$ per ogni $t \in I$;
            \item $H(s,0) = H(s,1)$ per ogni $s \in I$.
        \end{itemize}
        Si dice che $H$ è un'\textbf{omotopia di cammini a estremi fissi} e si scrive
        \[
            \gamma_0 \sim \gamma_1.
        \] 
        Infatti è facile verificare che l'essere omotopi a estremi fissi induce una relazione di equivalenza
        sull'insieme dei cammini in $X$.
    \end{definition}

    \begin{definition} [Giunzione di cammini] \nl
        Siano $f,g: I \to X$ due cammini in $X$ con $f(1) = g(0)$, allora la \textbf{giunzione} di $f$ e $g$ è il cammino
        \[
            f * g: I \to X: t \mapsto 
            \begin{cases}
                f(2t) & \text{se } 0 \leq t \leq \frac{1}{2}, \\
                g(2t-1) & \text{se } \frac{1}{2} < t \leq 1.
            \end{cases}
        \]
    \end{definition}

    \begin{lemma} [Giunzione di cammini e omotopia] \nl
        Se $f \sim f'$ e $g \sim g'$, allora $f * g \sim f' * g'$.
    \end{lemma}

    \begin{proof}
            Sia $H_f: I \times I \to X$ un'omotopia di $f$ e $f'$ e $H_g: I \times I \to X$ un'omotopia di $g$ e $g'$.

            Definiamo l'omotopia
            \[
                H: I \times I \to X: (s,t) \mapsto 
                \begin{cases}
                    H_f(2s,t) & \text{se } 0 \leq s \leq \frac{1}{2}, \\
                    H_g(2s-1,t) & \text{se } \frac{1}{2} < s \leq 1.
                \end{cases}
            \]
            che risulta continua. Infatti la continuità di $H_f$ e $H_g$ implica la continuità di $H$, essendo le due funzioni
            definite su due intervalli disgiunti. Inoltre si verifica facilmente che $H$ soddisfa le condizioni richieste.
    \end{proof}

    \begin{remark} \nl
        Si noti che la giunzione di cammini non è definita su ogni coppia di cammini, ma solo su quelle che hanno
        il punto finale del primo uguale al punto iniziale del secondo. Tuttavia, se si considerano solo i cammini chiusi \textbf{che partono da uno stesso punto iniziale},
        la giunzione è chiaramente sempre definita.
    \end{remark}

\subsection{Definizione del gruppo fondamentale}
    Da ora in poi gli spazi topologici considerati saranno sempre localmente connessi.

    \begin{theorem} [Poincar\'e] \nl
        Se $X$ uno spazio topologico e $x_0 \in X$ un punto fisso. \nl
        Il prodotto dato dalla giunzione di cammini induce una struttura di gruppo sulle classi di omotopia
        dei cammini chiusi in $X$ aventi punto iniziale $x_0$. \nl
        Tale gruppo è chiamato \textbf{gruppo fondamentale} di $X$ in $x_0$ e si denota con $\pi_1(X,x_0)$. \nl
        In tale gruppo l'elemento neutro è rappresentato dal cammino costante in $x_0$ e l'inverso di un cammino $\gamma$ è il cammino
        \[
            \gamma^{-1}(t) = \gamma(1-t)
        \]
        che è l'inverso rispetto alla giunzione di cammini.
    \end{theorem}

    Per la dimostrazione del teorema di Poincar\'e ci basta dimostrare prima un lemma.

    \begin{lemma} (Riparametrizzazione di un cammino e omotopia) \nl
        Sia $\gamma: I \to X$ un cammino in $X$ e sia $\varphi: I \to I$ una funzione continua tale che $\phi(0) = 0$ e $\phi(1) = 1$.
        Allora $\gamma \circ \varphi: I \to X$ è un cammino in $X$ e $\gamma \sim \gamma \circ \varphi$.
    \end{lemma}

    \begin{proof}
        Basta mostrare che la funzione $\varphi$ \`e omotopa all'identita\`a $id_I$. \nl
        L'omotopia \`e data dalla famiglia di funzioni
        \[
            \varphi_s: I \to I: t \mapsto (1-s) t + s \varphi(t).
        \]
        E poi boh.. buco.
    \end{proof}

    \begin{proof} [Teorema di Poincar\`e] \nl
        \begin{itemize}
            \item (Associatività) \nl
            Siano $\gamma_1, \gamma_2, \gamma_3: I \to X$ tre cammini chiusi in $X$ con punto iniziale $x_0$.
            Si ha che
            \[
                (\gamma_1 * \gamma_2) * \gamma_3 \sim \gamma_1 * (\gamma_2 * \gamma_3).
            \]
            Poich\'e $\gamma_1 * (\gamma_2 * \gamma_3)$ si pu\`o vedere come una Riparametrizzazione
            del cammino $(\gamma_1 * \gamma_2) * \gamma_3$ e quinid usare il lemma.
            \item (Unità) \nl
            L'elemento neutro del gruppo fondamentale è il cammino costante in $x_0$, che si denota con $e: I \to {x_0}$. \nl
            Infatti, per ogni cammino $\gamma: I \to X$ si ha che
            $\gamma * e $ \`e la Riparametrizzazione di $\gamma$ secondo la mappa
            \[
                \varphi: I \to I: t \mapsto \begin{cases}
                    2t & \text{se } 0 \leq t \leq \frac{1}{2}, \\
                    1 & \text{se } \frac{1}{2} < t \leq 1
                \end{cases}.
            \]
            \item (Inverso) \nl
            Sia 
            \[
                \gamma_s: I \to X: t \mapsto
                \begin{cases}
                    \gamma(t) & \text{se } 0 \leq t \leq s, \\
                    \gamma(s) & \text{se } s < t \leq 1.
                \end{cases}
            \]
            La famiglia di cammini $\set{\gamma_s}_{s \in I}$, che non sono lacci, rappresenta un'omotopia tra il cammino costante in $x_0$ e il cammino 
            $\gamma$, tuttavia \textbf{non rappresenta un'omotopia ad estremi fissi} poiché $\gamma_s(1) \neq \gamma(1)$.
            Vale pero\` che $\gamma_s(0) = \gamma(0)$ cio\`e il punto iniziale \`e fisso. \nl
            In modo analogo la famiglia di cammini data da 
            \[
                \gamma_s^{-1}(t) := gamma_s(1-t) 
            \]
            rappresenta un'omotopia tra il cammino costante in $x_0$ e il cammino $\gamma^{-1}$, ma non ad estremi fissi. \nl
            A questo punto si verifica che la famiglia di \textbf{cammini chiusi} $\set{\gamma_s * \gamma_s^{-1}}_{s \in I}$ rappresenta un'omotopia \textbf{ad estremi fissi} 
            tra il cammino costante in $x_0$ e il cammino $\gamma * \gamma^{-1}$. \nl
            Si fa in maniera analoga per mostrare che $\gamma^{-1} * \gamma \sim e_{x_0}$
        \end{itemize}
    \end{proof}

    \begin{example} \nl
        \[
            \pi_1(\R^n, x_0) = \{ e_{x_0} \} \quad \forall x_0 \in \R^n.
        \]

        Siano $\alpha, \beta: I \to \R^n$ due cammini chiusi in $\R^n$ con punto iniziale $x_0$. \nl
        La famiglia di cammini chiusi definita da
        \[
            f_s: I \to \R^n: t \mapsto (1-s) \alpha(t) + s \beta(t)
        \]
        definisce un'omotopia ad estremi fissi tra $\alpha$ e $\beta$. \nl
        Piu\` in generale, l'omotopia definita e\` quella che per ogni punto dei cammini percorre al variare di $s$
        il segmento che unisce i due cammmini in quell'istante $t$, e dunque la stessa argomentazione vale per dimostrare che:
       \[
        \begin{aligned}
            &\forall X \subset \R^n \text{ convesso}, \\
            &\pi_1(X, x_0) = \{ e_{x_0} \} \quad \forall x_0 \in \R^n.
        \end{aligned}
        \]
    \end{example}

    \begin{proposition} [Gruppo fondamentale di un connesso per archi] \nl
        Sia $X$ uno spazio topologico connesso per archi, allora
        \[
            \pi_1(X, x_0) \cong \pi_1(X, x_1) \quad \forall x_0, x_1 \in X.
        \]
        In altre parole, il gruppo fondamentale di uno spazio topologico connesso per archi non dipende dal punto iniziale scelto.
    \end{proposition}

    \begin{proof}
    Sia $f: I \to X$ un cammino tale che $f(0) = x_0$ e $f(1) = x_1$, che esiste poiché $X$ è connesso per archi.
    Tale cammino induce un isomorfismo tra i gruppi fondamentali in $x_0$ e $x_1$:
    \[
        \begin{aligned}
            \pi_1(X, x_0) &\xrightarrow{\sim} \pi_1(X, x_1) \\
            [\gamma] &\mapsto [f * \gamma * f^{-1}]
        \end{aligned}
    \]
    con inversa data da
    \[
        \begin{aligned}
            \pi_1(X, x_1) &\xrightarrow{\sim} \pi_1(X, x_0) \\
            [\gamma] &\mapsto [f^{-1} * \gamma * f].
        \end{aligned}
    \]
    Infatti, si verifica prima di tutto la buona definizione: \nl
    Se $\gamma_1 \sim \gamma_2$ sono due cammini chiusi in $X$, per il lemma della Riparametrizzazione, si ha che
    \[
        f * \gamma_1 * f^{-1} \sim f * \gamma_2 * f^{-1}.
    \]
    Inoltre, si verifica che l'immagine di un cammino chiuso in $x_0$ è un cammino chiuso in $x_1$ e viceversa. \nl
    Si si veririfica che le funzioni appena definite sono effettivamente degli omomorfismi di gruppo
    poiche\` si ha che: 
    \[
        f * \gamma_1 * \gamma_2 * f^{-1} \sim (f * \gamma_1 * f^{-1}) * (f * \gamma_2 * f^{-1})
    \]
    usando l'associatività che anche se non dimostrata vale anche per cammini chiusi. \nl
    Infine, si verifica facilmente che le due mappe sono una l'inversa dell'altra.
\end{proof}

\begin{remark} \nl
    L'isomorfismo tra i due gruppi fondamentali non è canonico, poiché dipende dalla scelta del cammino $f$ tra i due punti $x_0$ e $x_1$.
\end{remark}

\begin{definition} [Spazio semplicemente connesso] \nl
    Uno spazio topologico $X$ si dice \textbf{semplicemente connesso} se è connesso per archi e il suo gruppo fondamentale è banale, cioè
    \[
        \pi_1(X, x_0) = \{ e_{x_0} \} \quad \forall x_0 \in X.
    \]
\end{definition}

\begin{remark} \nl
    Se $X$ è semplicemente connesso e $\alpha, \beta: I \to X$ sono due cammini allora
    \[ \alpha(0) = \beta(0), \quad \alpha(1) = \beta(1) \implies \alpha \sim \beta\]
    Dato che il cammino $\alpha * \beta^{-1}$ è chiuso e il gruppo fondamentale è banale, quindi
    \[
        \alpha * \beta^{-1} \sim e_{x_0} \implies \alpha \sim \beta.
    \]
\end{remark}

\begin{remark} [La funtorialit\`a del gruppo fondamentale] \nl
    Siano $X, Y$ due spazi topologici e $\varphi: X \to Y$ una mappa continua tale che $\varphi(x_0) = y_0$ per due punti fissi $x_0 \in X$ e $y_0 \in Y$. \nl
    Allora $\varphi$ induce un omomorfismo di gruppi
    \[
        \varphi_*: \pi_1(X, x_0) \to \pi_1(Y, y_0)
    \]
    definito da
    \[
        \varphi_*([\gamma]) = [\varphi \circ \gamma]
    \]
    Si verifica facilmente che la mappa \`e ben definita ed \`e un omomorifsmo di gruppi. \nl
    Inoltre, vale che,
    se $\varphi = \id_X$ allora $\varphi_* = id_{\pi_1(X, x_0)}$ e se $(\psi \circ \varphi)_* = \psi_* \circ \varphi_*$. \nl
    Nel linguaggio delle categorie quindi si dice che 
    \[  
        \pi_1: \mathbf{Top} \to \mathbf{Grp}: X \mapsto \pi_1(X, x_0)
    \]
    è un \textbf{funtore} da $\mathbf{Top}$, la categoria degli spazi topologici, a $\mathbf{Grp}$, la categoria dei gruppi.
\end{remark}

\begin{proposition} \nl
    Se $\varphi: X \to Y$ è un omeomorfismo tra spazi topologici, allora
    \[
        \varphi_*: \pi_1(X, x_0) \to \pi_1(Y, y_0)
    \]
    è un isomorfismo di gruppi, dove $x_0 \in X$ e $y_0 = \varphi(x_0) \in Y$.
\end{proposition}

\begin{proof} \nl
    Poich\'e $\varphi$ è un omeomorfismo, essa è continua e ha un'inversa continua $\varphi^{-1}: Y \to X$. \nl
    Cio\`e  $\varphi^{-1} \circ \varphi = \id_X$ e $\varphi \circ \varphi^{-1} = \id_Y$, quindi segue dalla funtorialit\'a
    che
    \[
        \varphi_* \circ \varphi^{-1}_* = id_{\pi_1(X, x_0)} \quad \text{e} \quad \varphi^{-1}_* \circ \varphi_* = id_{\pi_1(Y, y_0)}.
    \]
    Quindi $\varphi_*$ è un isomorfismo di gruppi, poiché ha un'inversa data da $\varphi^{-1}_*$.
\end{proof}

\begin{definition} [Spazi omotopicamente equivalenti] \nl
    Due spazi topologici $X$ e $Y$ si dicono \textbf{omotopicamente equivalenti} se esistono due funzioni continue
    \[
        f: X \to Y \quad \text{e} \quad g: Y \to X
    \]
    tali che:
    \begin{itemize}
        \item $g \circ f$ è omotopa all'identità su $X$;
        \item $f \circ g$ è omotopa all'identità su $Y$.
    \end{itemize}
    Si denota con $X \simeq Y$ se $X$ e $Y$ sono omotopicamente equivalenti.
\end{definition}

\begin{example} \nl
    \begin{enumerate} 
        \item $\R^n$ \`e omotopicamente equivalente ad un punto, cio\`e si dice che $\R^n$ \`e \textbf{contraibile}.
            Infatti sia $\varphi: \R^n \to \set{0} \subset \R^n$ la funzione costante in $0$, che è continua.
            e sia $\psi: \set{0} \to \R^n$ anch'essa continua. \nl
            Si ha che $\varphi \circ \psi = \id_{\set{0}}$, mentre $\psi \circ \varphi$ è omotopa all'identità su $\R^n$ tramite l'omotopia
            definita da
            \[
                H: I \times \R^n \to \R^n: (s,x) \mapsto s x.
            \]
        \item $S^n$ \`e omotopicamente equivalente a $\R^{n+1} \setminus{\set{0}}$ \nl
        Infatti se $i: S^n \into \R^n$ \`e l'inclusione di $S^n$ in $\R^{n+1} \setminus{\set{0}}$ e
        \[  
            \psi: \R^{n+1} \setminus{\set{0}} \to S^n: x \mapsto \frac{x}{\norm{x}}.
        \]
        Si ha che $i \circ \psi = \id_{S^n}$ e $\psi \circ i \sim \id_{\R^{n+1} \setminus{\set{0}}}$ tramite l'omotopia
        \[
            H: I \times \R^{n+1} \setminus{\set{0}} \to \R^{n+1} \setminus{\set{0}}: (s,x) \mapsto (1-s)x + s\frac{x}{\norm{x}}.
        \]
        \item Il Nastro di Möbius \`e omotopicamente equivalente al cerchio $\Sph{1}$. \nl
            Infatti, sia $M$ il Nastro di Möbius e sia $\varphi: M \to \Sph{1}$ la proiezione che manda ogni punto del nastro sul suo bordo.
            Si ha che $\varphi$ è continua e suriettiva. \nl
            Infatti se consideriamo il quadrato $Q = [-1,1] \times [-1,1]$, tale spazio \`e omotopicamente equivalente al segmento
            $[-1,1]$ tramite l'inlcusione del segmento nel quadrato e la proiezione naturale del quadrato sul segmento.
            Identificando i lati opposti del quadrato in modo da ottenere il Nastro di Möbius, si ha che
            la proiezione del quadrato sul segmento induce una mappa continua e suriettiva dal Nastro di Möbius al cerchio, con omotopie 
            che passano al quoziente.
        \end{enumerate}
\end{example}

\begin{theorem} (Spazi omotopicaamente equivalenti hanno gruppo fondamentale isomorfo) \nl
    Siano $X$ e $Y$ spazi topologici \textbf{connessi per archi} omotopicamente equivalenti, allora i loro gruppi fondamentali sono isomorfi:
    \[
        \pi_1(X, x_0) \cong \pi_1(Y, y_0)
    \]
    per ogni coppia di punti fissi $x_0 \in X$ e $y_0 \in Y$.
\end{theorem}

\begin{lemma} \nl
    Siano $\varphi_0, \varphi_1: X \to Y$ due funzioni continue \textbf{omotope} tra spazi topologici e siano $x_0 \in X$. \nl
    Il seguente diagramma commuta:
    % https://q.uiver.app/#q=WzAsMyxbMCwxLCJcXHBpXzEoWF8seF8wKSJdLFsxLDAsIlxccGlmKFNfMSxYLFxcdmFycGhpXzAoeF8wKSkiXSxbMSwyLCJcXHBpXzEoWSxcXHZhcnBoaV8xKHhfMCkpIl0sWzAsMSwiXFx2YXJwaGlfezAqfSJdLFswLDIsIlxcdmFycGhpX3sxKn0iLDJdLFsxLDIsIlxcdGF1X2YiXV0=
    \[
        \begin{tikzcd}
        & {\pi_1\left(Y,\varphi_0\left(x_0\right)\right)} \\
        {\pi_1\left(X_,x_0\right)} \\
        & {\pi_1\left(Y,\varphi_1\left(x_0\right)\right)}
        \arrow["{\tau_f}", from=1-2, to=3-2]
        \arrow["{\varphi_{0*}}", from=2-1, to=1-2]
        \arrow["{\varphi_{1*}}"', from=2-1, to=3-2]
        \end{tikzcd}
    \]
    dove $\tau_f: \pi_1\left(Y, \varphi_0\left(x_0\right)\right) \to \pi_1\left(Y, \varphi_1\left(x_0\right)\right)$ è l'isomorfismo indotto dal cammino
    $f: I \to Y: s \mapsto \varphi_s\left(x_0\right)$ e $\set{\varphi_s \mid s \in I}$ \`e l'omotopia tra $\varphi_0$ e $\varphi_1$.
\end{lemma}

\begin{proof} \nl
    Si consideri la mappa
    \[
        \tau^{-1}_f := \tau_{f^{-1}}: \pi_1\left(Y, \varphi_1\left(x_0\right)\right) \to \pi_1\left(Y, \varphi_0\left(x_0\right)\right): g_Y \mapsto f * g_Y * f^{-1}.
    \]
    Al variare di $s \in I$ si ha che
    \[
        f_s: I \to Y: t \mapsto f\left(st\right)
    \]
   rappresenta un'omotopia tra il cammino $f_0: I \to \set{\varphi_0\left(x_0\right)}$ 
   e il cammino $f$. \nl
   Quindi, se ora si considera $g_X$ un cammino chiuso in $x_0 \in X$, allora la mappa
   \[
        I \to \pi_1\left(Y, \varphi_0\left(x_0\right)\right): s \mapsto f_s * \varphi_{0}\left(g_X\right) * f_s^{-1}
   \]
   induce un'omotopia tra il cammino chiuso $\varphi_{0}\left(g_X\right)$ e il cammino chiuso
    $f\left(\varphi_{1}\left(g_X\right)\right)$, dunque vale  che
    \[
        \varphi_{0*}\left(g_X\right) = \tau_f\left(\varphi_{1*}\left(g_X\right)\right).
    \]
\end{proof}

\begin{proof} [del teorema] \nl
    Siano $\varphi: X \to Y$ e $\psi: Y \to X$ le funzioni continue che definiscono l'equivalenza omotopica tra $X$ e $Y$. \nl
    Grazie al lemma precedenta, dato che vale $\psi \circ \varphi \sim \id$ si ha che il seguente diagramma commuta:
    % https://q.uiver.app/#q=WzAsNCxbMCwwLCJcXHBpXzFcXGxlZnQoWCwgeF8wIFxccmlnaHQpIl0sWzEsMCwiXFxwaV8xXFxsZWZ0KFksIFxcdmFycGhpXFxsZWZ0KHhfMFxccmlnaHQpIFxccmlnaHQpIl0sWzIsMCwiXFxwaV8xXFxsZWZ0KFksIChcXHBzaSBcXGNpcmMgXFx2YXJwaGkpXFxsZWZ0KHhfMFxccmlnaHQpIFxccmlnaHQpIl0sWzIsMSwiXFxwaV8xXFxsZWZ0KFgsIHhfMCBcXHJpZ2h0KSJdLFswLDMsIlxcaWQiLDJdLFswLDEsIlxcdmFycGhpXyoiXSxbMSwyLCJcXHBzaV8qIl0sWzIsMywiXFx0YXVfZiJdXQ==
    \[
    \begin{tikzcd}
        {\pi_1\left(X, x_0 \right)} & {\pi_1\left(Y, \varphi\left(x_0\right) \right)} & {\pi_1\left(X, (\psi \circ \varphi)\left(x_0\right) \right)} \\
        && {\pi_1\left(X, x_0 \right)}
        \arrow["{\varphi_*}", from=1-1, to=1-2]
        \arrow["\id"', from=1-1, to=2-3]
        \arrow["{\psi_*}", from=1-2, to=1-3]
        \arrow["{\tau_f}" right, "{\cong}" left, from=1-3, to=2-3]
    \end{tikzcd}
    \]
    Cio\`e vale che $ \tau_f \circ \psi_* \circ \varphi_* = \id$, quindi $\psi_* \circ \varphi_* = \tau_f^{-1}$, ma se la composizione di due mappe
    \`e bigettiva allora la prima $\varphi_*$ \`e iniettiva e la seconda $\psi_*$ \`e suriettiva, ragionando in maniera analoga
    per il verso opposto si ha che $\varphi_* \circ \psi_* = \tau_f$ e quindi $\psi_*$ \`e iniettiva e $\varphi_*$ \`e surgettiva. \nl
    Si conclude quindi che $\varphi_*$ e $\psi_*$ sono isomorfismi di gruppi.
         
\end{proof}

\subsection{Primi gruppi fondamentali}

Da questo momento in poi, se $X$ \`e uno spazio topologico connesso per archi, si denota con $\pi_1(X)$ il gruppo fondamentale di uno spazio topologico $X$ in un punto fissato.

\begin{theorem} [Gruppo fondamentale del cerchio] \nl
    \[
        \pi_1\left(\Sph{1} \right) \cong \Z.
    \]
    Ed il cammino chiuso $t \mapsto e^{2\pi i t}$ rappresenta il generatore del gruppo fondamentale $\pi_1\left(\Sph{1}, 1\right)$.
\end{theorem}

\begin{definition} [Mappa esponenziale] \nl
    Si definisce la mappa
    \[
        \rho: \R \to \Sph{1}: t \mapsto e^{2\pi i t}
    \]
\end{definition}

\begin{lemma} [Sollevamento di un cammino di $\Sph{1}$ in $\R$] \nl
    \begin{enumerate}
        \item Per ogni cammino chiuso $f: I \to \Sph{1}$ con $f(0) = f(1)$, 
        \textbf{esiste ed unico} un cammino(in generale non chiuso) $\tilde{f}: I \to \R$ detto \textbf{sollevamento di $f$ in $\R$} tale che
        \[
            \tilde{f}(0) = 0 \quad \text{e} \quad \rho \circ \tilde{f} = f.
        \]Il terminale mi dice di aver committato, ma su github la repository non sembra essere commi
        Ovvero il seguente diagramma commuta:
        \[\begin{tikzcd}
            I & \R \\
            & {\Sph{1}}
            \arrow["{\tilde{f}}", from=1-1, to=1-2]
            \arrow["f"', from=1-1, to=2-2]
            \arrow["\rho", from=1-2, to=2-2]
        \end{tikzcd}\]
        \item Inoltre se $f_0, f_1$ sono due cammini chiusi omotopi allora
        \[
            \tilde{f}_0(1) = \tilde{f}_1(1) \in \Z
        \] 
    \end{enumerate}
\end{lemma}

\begin{proof} (Lemma $\implies$ Teorema) \nl
    Dal lemma segue che la mappa:
    \[
        \Phi: \pi_1\left(\Sph{1}, 1\right) \to \Z: [f] \mapsto \tilde{f}(1)
    \]Il terminale mi dice di aver committato, ma su github la repository non sembra essere commi
    \`e ben definita, ed inoltre induce un omomorfismo di gruppi, poich\'e
    \[
        \Phi\left([\gamma_1 * \gamma_2]\right) = \tilde{\gamma}_1(1) + \tilde{\gamma}_2(1) = \Phi\left([\gamma_1]\right) + \Phi\left([\gamma_2]\right).
    \].
    Si dimostra ora la surgettivit\`a di $\Phi$, infatti dato il cammino chiuso $f_1: I \to \Sph{1}: t \mapsto e^{2\pi i t}$, si ha che
    \[
        \Phi\left([f_1^n]\right) = n \tilde{f}_1(1) = n.
    \]
    Infine, si verifica che il nucleo di $\Phi$ \`e l'insieme dei cammini chiusi omotopi al cammino costante in $1$, dato che
    se $f: I \to \Sph{1}$ \`e un cammino chiuso tale che $\tilde{f}(1) = 0$, si ha che $\tilde{f}$ \`e un cammino chiuso in $\R$ che parte da $0$ e torna a $0$, quindi
    poich\'e $\R$ \`e semplicemente connesso, esiste un'omotopia $H: I \times I \to \R$ da $\tilde{f}$ al camminio costante in $0$.
    Ma a questo punto si ha che $\rho \circ H$ è un'omotopia da $f$ al cammino costante in $1$, quindi $f$ è omotopo al cammino costante in $1$.
\end{proof}

\begin{proof} (del teorema) \nl
    OK LA DIMOSTRAZIONE DI QUESTO FATTO FATTA DA TAMAS \`E RIDICOLA, MEGLIO FARE QUELLA PIU\' GENERALE DI FRIGERIO QUANDO SAR\`A POI
    \begin{enumerate}
        \item si consideri il ricoprimento di aperti di $\Sph{1}$ dato da due aperti $U_0, U_1$, archi che si intersecano in due componenti 
        connesse per archi di $\Sph{1}$, una che contiene $1$ e l'altra che contiene $-1$. 
        \nl
        DISEGNO DA FARE
        \nl
        Considero le componenti connesse per archi di $\rho^{-1}(U_1)$, che formano un ricorpimento di aperti
        per $\rho^{-1}(U_1)$. La mappa $\rho$ induce un omeomorfismo 
        $\rho\big|_{\rho^{-1}(U_1)}: \rho^{-1}(U_1) \to U_1$(si vedr\`a che \`e un rivestimento di $U_1$)).
        \nl
        Inoltre, le componenti connesse per archi di $f^{-1}(U_0), f^{-1}(U_1)$ formano un ricoprimento di aperti per $I$.
        \nl
        L'intervallo $I$ \`e uno spazio metrico compatto, e dunque ammette un numero di Lebesgue. \nl
        Siano quindi $t_0 = 0 < t_1 < t_2 < \ldots < t_n = 1$ i punti di $I$ tali che
        ciascun intervallo della $\left[t_i, t_{i+1}\right]$ \`e interamente contenuto in uno
        ed uno solo tra  $f^{-1}(U_0)$ e $f^{-1}(U_1)$ e inoltre $t_i \in f^{-1}(U_0) \cap f^{-1}(U_1) \quad \forall i$.
        \nl
    \end{enumerate}
\end{proof}

\begin{corollary} \nl
    \[  
        \pi_1(\R^n \setminus \set{0}) \cong \pi_1(D^n \setminus \set{0}) \cong \Z
    \]
    In particolare $\Disc{2} \setminus \set{0}$ non \`e omotopicamente equivalente $\Disc{2}$(e quindi nemmeno omeomorfo).
\end{corollary}

\begin{definition} [Retrazione] \nl
    Sia $Y \subset X$ un sottospazio topologico. Si dice che una mappa continua $r: X \to Y$ \`e una retraazione se vale
    \[
        r \circ i = \id_Y
    \]
    dove $i: Y \into X$ \`e l'inclusione di $Y$ in $X$. \nl 
    In altre parole, $r$ \`e una retrazione se \`e continua e manda ogni punto di $Y$ su se stesso. \nl
    Si dice che $Y$ \`e \textbf{retratto} in $X$ se esiste una retrazione da $X$ a $Y$.
\end{definition}

\begin{example}
    \begin{enumerate}
        \item  In ogni spazio topologico $X$ ogni punto ${x_0} \subset X$ \`e un retratto di $X$.
        \item  Il segmento $I = [-1, 1]$ \`e un retratto di $\bar{\Disc{2}} = \closedball{0}{1}$, infatti la mappa
            \[
                r: \bar{\Disc{2}} \to I: (x,y) \mapsto x
            \]
            \`e una retrazione, poich\'e $r$ manda ogni punto del segmento su se stesso.

    \end{enumerate}
\end{example}

\begin{lemma} [Retrazione e gruppo fondamentale] \nl
    Sia $Y \subset X$ un retratto di $X$ e sia $x_0 \in Y$. Allora la mappa indotta dall'inclusione naturale
    \[
        i_*: \pi_1(Y, x_0) \to \pi_1(X, x_0)
    \]
    \`e un omomorfismo di gruppi \textbf{iniettivo}
\end{lemma}

\begin{proof} \nl
    \[\begin{tikzcd}
	{\pi_1\left(Y, x_0\right)} & {\pi_1\left(X, x_0\right)} & {\pi_1\left(Y, x_0\right)}
	\arrow["{i_*}", from=1-1, to=1-2]
	\arrow["\id"', curve={height=18pt}, from=1-1, to=1-3]
	\arrow["{r_*}", from=1-2, to=1-3]
    \end{tikzcd}\]
    Dunque $r_* \circ i_* = \id_{\pi_1(Y, x_0)}$, quindi $i_*$ \`e iniettiva perch\'e ha inversa sinistra. \nl
\end{proof}

\begin{corollary} \nl
    $\Sph{1}$ non \`e un retratto di $\bar{\Disc{2}}$.
\end{corollary}

\begin{proof}
    \[\begin{tikzcd}
	\Z & {\pi_1\left(\Sph{1}, 1\right)} & {\pi_1\left(\bar{\Disc{1}}, 1\right)} & {\set{1}}
	\arrow["\cong"{description}, draw=none, from=1-1, to=1-2]
	\arrow["{i_*}", from=1-2, to=1-3]
	\arrow["\cong"{description}, draw=none, from=1-3, to=1-4]
\end{tikzcd}\]
    e quindi $i_*$ non \`e iniettiva, perch\'e \`e forzata ad essere banale. 
\end{proof}

\begin{theorem} [Brouwer] \nl
    Ogni applicazione continua $f: \Disc{2} \to \Disc{2}$ ammette un punto fisso.
\end{theorem}

\begin{proof}
    La dimostrazione \`e per assurdo. \nl
    Si supponga che per ogni punto $x \in \Disc{2}$ si ha che $f(x) \neq x$. \nl
    Consideriamo la mappa continua
    \[
        r: \Disc{2} \to \Sph{1}: x \mapsto \frac{f(x) - x}{\norm{f(x) - x}}.
    \]
    Questa mappa associa ad ogni punto $x \in \Disc{2}$ un punto sulla circonferenza unitaria $\Sph{1}$, che rappresenta la direzione del vettore che punta da $x$ a $f(x)$. \nl
    Una tale mappa sarebbe una retrazione del disco unitario $\Disc{2}$ su $\Sph{1}$, poiché ogni punto di $\Sph{1}$ sarebbe raggiunto da un punto di $\Disc{2}$ che non si mappa su se stesso. 
    $\contradiction$
\end{proof}

\begin{theorem} [Fondamentale dell'algebra] \nl
    Ogni $f(x) \in \C[x]$ polinomio di grado $n \geq 1$ ammette almeno una radice complessa.
\end{theorem}

\begin{proof} \nl
    Si supponga $f(z) \neq 0 \quad \forall z \in \C$, allora $\forall r > 0$ la mappa
    \[
        f_r(t) := \frac{f\left(r \cos(2\pi t) + ir\sin(2 \pi t)\right)}{\abs{f\left(r \cos(2\pi t) + ir\sin(2 \pi t)\right)}}
    \]
    definisce un cammino chiuso in $I \to \Sph{1}$ che parte da $1$. \nl
    La famiglia di cammini chiusi $\set{f_r | r \in I}$ rappresenta un'omotopia tra il cammino costante $f_0: I \to \set{1}$ e il cammino chiuso
    $f_1$. \nl
    Componendo inoltre con la mappa $I \to [0,r]: s \to sr$ otteniamo un'omotopia ad estremi fissi tra il cammino costante e il cammino chiuso $f_r$. \nl
    Quindi $\left[f_r^0 \right] = 0 \in \Z$. Vogliamo ora dimostrare che $\left[f_r^1 \right] \neq 0 \in \Z$  per ogni $r > 0$ e raggiungere una contraddizione. \nl
    Sia ora il polinomio
    \[
        f(z) = z^n + a_{n-1}z^{n-1} + \ldots + a_0
    \]
    e per $s \in I$ si consideri
    \[
        f_r^s(z) = z^n + s\left( a_{n-1}z^{n-1} + \ldots + a_0 \right).
    \]
    Se $r > \max\set{1, \sum \abs{a_i}}$ e $\abs{z} = r$, allora
    \[
        \abs{z^n} = r^n > s \left( \sum \abs{a_i} \right) \abs{z^{n-1}} \geq \abs{s\left(a_{n-1}z^{n-1} + \dots + a_0\right)}
    \]
    e quindi poich\'e vi \`e il maggiore stretto se $\abs{z} = r$, $f_r^s(z) \neq 0$ per ogni $s \in I$. \nl
    In particolare vale 
    \[
        f_r^s(r \cos{2\pi t} + ir\sin(2 \pi t) \neq 0 
    \]
    E quindi \`e ben definita la famiglia di cammini chiusi
    \[
        f_r^s: I \to \Sph{1}: t \mapsto \frac{f_r^s\left(r \cos{2\pi t} + ir\sin(2 \pi t)\right)}{\abs{f_r^s\left(r \cos{2\pi t} + ir\sin(2 \pi t)\right)}}.
    \]
    che da un'omotopia tra $f_r^0$ e $f_r^1$. \nl
    $f^0 = z^n$ e quindi $f_r^0(t) = \cos(2 n \pi t) + i\sin(2 n \pi t)$
    ma si avrebbe quindi che la classe di omotopia di $f_r^0$ \`e $n \in \Z$, ma ci\`o contraddice il fatto che
    $\left[f_r^0 \right] = 0 \in \Z$.
\end{proof}

\subsection{Teorema di Seifert-Van Kampen e richiami di teoria dei gruppi}

\begin{theorem} [debole di Seifert-van Kampen] \nl
    Siano $X = X_1 \cup X_2$ dove $X_1, X_2 \subset X$ sono aperti. Siano $i_1: X_1 \into X$ e $i_2: X_2 \into X$ le inclusioni naturali. \nl
    Si suppongano $X, X_1, X_2, X_1 \cap X_2$ connessi per archi allora:
    \[
        \pi_1(X, x_0) \text{\`e generato da } i_{1*}\left(\pi_1\left(X_1, x_0\right)\right) \text{ e } i_{2*}\left(\pi_1\left(X_2, x_0\right)\right) \text{dove } x_0 \in X_1 \cap X_2.
    \]
\end{theorem}

\begin{proof}

\end{proof}

\begin{corollary} \nl
    $\Sph{n}$ \`e semplicemente connesso per ogni $n \geq 1$.'
\end{corollary}

\begin{corollary} \nl
    $\R^n \setminus \set{0}$ \`e omotopicamente equivalente a $\Sph{n-1}$ per ogni $n \geq 2$. \nl
    Quindi $\pi_1(\R^2 \setminus \set{0}) \cong \Z$ e $\pi_1(\R^n \setminus \set{0}) \cong \set{1}$ per ogni $n \geq 2$
\end{corollary}

\begin{corollary} \nl
    $\R^2$ non \`e omeomorfo a $\R^n$ per ogni $n \geq 3$.
\end{corollary}

\begin{theorem} (di Seifert-van Kampen) \nl
    Siano $X = X_1 \cup X_2$ dove $X_1, X_2 \subset X$ sono aperti. \nl
    Siano $j_1: X_1 \cap X_2 \into X_1, j_2, X_1 \cap X_2 \into X_2, i_1: X_1 \into X, i_2: X_2 \into X$ le inclusioni naturali. \nl
    Si suppongano $X, X_1, X_2, X_1 \cap X_2$ connessi per archi allora

    $\forall G$ gruppo, e mappe $\varphi_1: \pi_1(X_1 \cap X_2, x_0) \to G$ e $\varphi_2: \pi_1(X_2, x_0) \to G$ esiste un unica mappa
    \[
        \varphi: \pi_1(X, x_0) \to G
    \]
    tale che il seguente diagramma commuta:
    % https://q.uiver.app/#q=WzAsNSxbMSwwLCJcXHBpXzFcXGxlZnQoWF8xLCB4XzBcXHJpZ2h0KSJdLFsxLDIsIlxccGlfMVxcbGVmdChYXzIsIHhfMFxccmlnaHQpIl0sWzAsMSwiXFxwaV8xXFxsZWZ0KFhfMSBcXGNhcCBYXzIsIHhfMFxccmlnaHQpIl0sWzIsMSwiXFxwaV8xXFxsZWZ0KFgsIHhfMFxccmlnaHQpIl0sWzMsMSwiRyJdLFswLDQsIlxcdmFycGhpXzEiXSxbMCwzLCJpX3sxKn0iLDJdLFsxLDMsImlfezIqfSJdLFszLDQsIlxcZXhpc3RzIVxcdmFycGhpIiwwLHsiY29sb3VyIjpbMCw2MCw2MF19LFswLDYwLDYwLDFdXSxbMiwwLCJqX3sxKn0iXSxbMiwxLCJqX3syKn0iLDFdLFsxLDQsIlxcdmFycGhpXzIiXV0=
    \[\begin{tikzcd}
	& {\pi_1\left(X_1, x_0\right)} \\
	{\pi_1\left(X_1 \cap X_2, x_0\right)} && {\pi_1\left(X, x_0\right)} & G \\
	& {\pi_1\left(X_2, x_0\right)}
	\arrow["{i_{1*}}"', from=1-2, to=2-3]
	\arrow["{\varphi_1}", from=1-2, to=2-4]
	\arrow["{j_{1*}}", from=2-1, to=1-2]
	\arrow["{j_{2*}}"{description}, from=2-1, to=3-2]
	\arrow["{\exists!\varphi}", color={rgb,255:red,214;green,92;blue,92}, from=2-3, to=2-4]
	\arrow["{i_{2*}}", from=3-2, to=2-3]
	\arrow["{\varphi_2}", from=3-2, to=2-4]
    \end{tikzcd}\]
\end{theorem}

Il teorema di Van-Kampen necessit\`a di un richiamo di teoria dei gruppi, che non \`e stato ancora fatto. \nl

\begin{definition} [Gruppo libero generato da un insieme] \nl
    Dato un insieme $X$ si indica $F(X)$ il \textbf{gruppo libero generato da $X$} il dato di
    un gruppo $F(X)$ ed una mappa iniettiva $s: X \into F(X)$ tale che la seguente propriet
    `a universale sia soddisfatta: \nl 
    Per ogni gruppo $G$ e per ogni mappa iniettiva $f: X \into G$ esiste un unico omomorfismo di gruppi
    \[
        \bar{f}: F(X) \to G
    \]
    tale che il seguente diagramma commuta:

    \[\begin{tikzcd}
	{F(X)} & G & {} \\
	X
	\arrow["{\exists!\bar{f}}", from=1-1, to=1-2]
	\arrow["s", from=2-1, to=1-1]
	\arrow["f"', from=2-1, to=1-2]
    \end{tikzcd}\]

\end{definition}

\begin{proposition} [Unicit\`a del gruppo libero] \nl
    Dalla definizione di gruppo libero via propriet\`a universale ne segue l'unicit\`a a meno di isomorfismo di gruppi.
\end{proposition}

\begin{proof}
    Facile provaci un attimo
\end{proof}

\begin{proposition} [Costruzione del gruppo libero generato da un insieme] \nl
    Si costruisce ora $F(X)$ nel seguente modo: \nl
    Sull'insieme
    \[
        F(X) = \set{w \in X^* \mid w \text{ parola su } X} / \sim
    \]
    dove una parola $w \in X$ \`e una sequenza un prodotto formale tra simboli della fomra
    \[
        w = x_{i_1}^{\epsilon_1} x_{i_2}^{\epsilon_2} \ldots x_{i_n}^{\epsilon_n}
    \]
    con $x_i \in X$ e $\epsilon_i \in \set{1, -1}$, e la relazione di equivalenza $\sim$ identifica due parole
    se e solo se sono uguali a meno di semplificare i fattori di forma $x_i^{\epsilon_i} x_i^{-\epsilon_i}$. \nl
    L'operazione di gruppo su $F(X)$ \`e data dalla concatenazione formale di parole. \nl

    Tale costruzione verifica la propriet\`a universale del gruppo libero generato da $X$.
\end{proposition}

\begin{proof} \nl
    Per ogni mappa $f: X \into G$ in un gruppo $G$ si definisce la mappa
    \[
        \bar{f}: F(X) \to G: x_{i_1}^{\epsilon_1} x_{i_2}^{\epsilon_2} \ldots x_{i_n}^{\epsilon_n} \mapsto f(x_1)^{\epsilon_1} f(x_2)^{\epsilon_2} \ldots f(x_n)^{\epsilon_n}
    \]
\end{proof}

\begin{lemma} \nl
    Ogni gruppo $G$ \`e il quoziente di un gruppo libero.
\end{lemma}

\begin{proof} \nl
    Se $\set{g_i \mid i \in I}$ si considera $X = \set{x_i \mid i \in I}$ e la mappa
    \[
        \Phi: F(X) \to G: x_i \to g_i \quad \text{ assegnamento per generatori}
    \]
    se i $g_i$ sono un insieme di generatori per $G$ allora $\Phi$ \`e surgettiva e si conclude
    per il primo teorema di omomorfismo tra gruppi.
\end{proof}

\begin{definition} [Presentazione di un gruppo] \nl
    Data $\Phi$ come sopra, sia $N := \ker{\Phi}$ si pu\`o considerare un \textbf{sisterma di generatori per N come sottogruppo normale}
    $\set{p_j \mid j \in J}$, la \textbf{presentazione tramite generatori e relazioni di G} \`e la seguente:
    \[
        G := \left< g_i, \ i \in I \mid p_j, \ j \in J \right>
    \]
\end{definition}

\begin{definition} [Prodotto libero di gruppi] \nl
    Siano $G_1, G_2$ due gruppi, si definisce il \textbf{prodotto libero di gruppi} $G_1 * G_2$
    il dato di un gruppo $G_1*G_2$ e mappe $\gamma_1: G_1 \to _1*G_2$, $\gamma_2: G_2 \to _1*G_2$
    che soddisfano la seguente propriet\`a universale: \nl
    Per ogni altro gruppo $G$ e mappe $\phi_1: G_1 \to G$ e $\phi_2: G_2 \to G$
    esiste un'unica mappa $\phi: G_1 * G_2 \to G$ tale che il seguente diagramma commuti:
     \[
    \begin{tikzcd}
    {G_1} \\
    & {G_1*G_2} & {\textcolor{blue}{G}} \\
    {G_2}
    \arrow["{\gamma_1}", from=1-1, to=2-2]
    \arrow["{\varphi_1}", color={rgb,255:red,92;green,92;blue,214}, curve={height=-12pt}, from=1-1, to=2-3]
    \arrow["{\exists!\varphi}", color={rgb,255:red,214;green,92;blue,92}, dashed, from=2-2, to=2-3]
    \arrow["{\gamma_2}"', from=3-1, to=2-2]
    \arrow["{\varphi_2}", color={rgb,255:red,92;green,92;blue,214}, curve={height=12pt}, from=3-1, to=2-3]
    \end{tikzcd}
    \]
    In teoria delle categorie tale costruzione \`e detta \textbf{coprodotto}.
\end{definition}

\begin{proposition} [Unicit\`a del prodotto libero di gruppi] \nl
    Dalla definizione via propriet\`a universale segue che il prodotto libero \`e unico a meno di isomorfismo di gruppi.
\end{proposition}

\begin{proof}
    Da fare, facile
\end{proof}

\begin{proposition} [Costruzione del prodotto libero tra gruppi]
    Siano 
    \[
        G_1 = \left< g_i^1, \ i \in I_1 \mid p_j^1, \ j \in J_1 \right> \quad G_2 = \left< g_i^2, \ i \in I_2 \mid p_j^2, \ j \in J_2 \right>
    \]
    le due presentazioni dei gruppi, allora la presentazione del prodotto libero \`e data da:
    \[
        G_1*G_2 = \left< \set{g_i^1} \cup \set{g_i^2} \bigm| \set{p_j^1} \cup \set{p_j^2} \right>
    \]

\end{proposition}

\begin{remark} \nl
    \begin{enumerate}
        \item Il gruppo libero \`e generato dalle immagini dei generatori dei gruppi fattori.
        \item Gli elementi di $G_1*G_2$ sono parole in $G_1 \cup G_2$
        \item Se $X_1, X_2$ sono due insiemi allora 
            \[
                F(X_1 \cup X_2) = F(X_1) * F(X_2)
            \]
    \end{enumerate}
\end{remark}

\begin{definition} [Prodotto amalgamato di gruppi] \nl
    Siano $G_1, G_2$ due gruppi e $H$ un terzo gruppo, si definisce il \textbf{prodotto amalgmato di gruppi su H} $G_1 *_H G_2$
    il dato di un gruppo $G_1 *_H G_2$ e mappe $\beta_1: H \to G_1$, $\beta_2: H \to G_2, \gamma_1: G_1 \to _1*G_2$, $\gamma_2: G_2 \to _1*G_2$
    che soddisfano la seguente propriet\`a universale: \nl
    Per ogni altro gruppo $G$ e mappe $\phi_1: G_1 \to G$ e $\phi_2: G_2 \to G$
    esiste un'unica mappa $\phi: G_1 * G_2 \to G$ tale che il seguente diagramma commuti:
    % https://q.uiver.app/#q=WzAsNSxbMSwyLCJHXzIiXSxbMSwwLCJHXzEiXSxbMiwxLCJHXzEqR18yIl0sWzMsMSwiRyJdLFswLDEsIkgiXSxbMSwyLCJcXGdhbW1hXzEiXSxbMCwyLCJcXGdhbW1hXzIiLDJdLFsxLDMsIlxcdmFycGhpXzEiLDAseyJjdXJ2ZSI6LTIsImNvbG91ciI6WzI0MCw2MCw2MF19LFsyNDAsNjAsNjAsMV1dLFswLDMsIlxcdmFycGhpXzIiLDAseyJjdXJ2ZSI6MiwiY29sb3VyIjpbMjQwLDYwLDYwXX0sWzI0MCw2MCw2MCwxXV0sWzIsMywiXFxleGlzdHMhXFx2YXJwaGkiLDAseyJjb2xvdXIiOlswLDYwLDYwXSwic3R5bGUiOnsiYm9keSI6eyJuYW1lIjoiZGFzaGVkIn19fSxbMCw2MCw2MCwxXV0sWzQsMSwiXFxiZXRhXzEiXSxbNCwwLCJcXGJldGFfMiJdXQ==
    \[\begin{tikzcd}
        & {G_1} \\
        H && {G_1*G_2} & G \\
        & {G_2}
        \arrow["{\gamma_1}", from=1-2, to=2-3]
        \arrow["{\varphi_1}", color={rgb,255:red,92;green,92;blue,214}, curve={height=-12pt}, from=1-2, to=2-4]
        \arrow["{\beta_1}", from=2-1, to=1-2]
        \arrow["{\beta_2}", from=2-1, to=3-2]
        \arrow["{\exists!\varphi}", color={rgb,255:red,214;green,92;blue,92}, dashed, from=2-3, to=2-4]
        \arrow["{\gamma_2}"', from=3-2, to=2-3]
        \arrow["{\varphi_2}", color={rgb,255:red,92;green,92;blue,214}, curve={height=12pt}, from=3-2, to=2-4]
    \end{tikzcd}\]
\end{definition}

\begin{proposition} [Costruzione del prodotto amalgamato se $H$ \`e un gruppo libero] \nl
    Siano
    \[
        G_1 = \left< g_i^1, \ i \in I_1 \mid p_j^1, \ j \in J_1 \right> \quad G_2 = \left< g_i^2, \ i \in I_2 \mid p_j^2, \ j \in J_2 \right>
    \]
    \[
        H =  \left< g_i^1, \ i \in I_1 \right> \text{ che non ha relazioni perch\'e libero}
    \]
    allora
    \[
        G_1 *_H G_2 = \left< \set{g_i^1} \cup \set{g_i^2} \bigm| \set{p_j^1} \cup \set{p_j^2} \cup \set{\beta_1(h_i)\beta_2(h_i)^{-1}} \right>
    \]
\end{proposition}

\begin{proposition} [Prodotto amalgamato di gruppi nel caso in cui uno dei fattori \`e banale] \nl
    
\end{proposition}

\begin{proposition} [Van-Kampen visto come prodotto amalgamato di gruppi ] \nl
    
\end{proposition}

\subsection{Applicazioni ed esempi}

\begin{enumerate}
    \item Calcolo del gruppo fondamentale del disco senza due punti:
            \[  
                \Disc{2} \setminus \set{x_0, x_1}
            \]
        Si scrive $\Disc{2}  \setminus \set{x_1, x_2} = X_1 \cap X_2$, dove $X_1, X_2$ sono due aperti di $\Disc{2}$ e
        l'intersezione $X_1 \cap X_2$ \`e semplicemente connessa. \nl
        Inoltre $X_i \cong \Disc{2} \setminus \set{x_i}$ per $i = 1, 2$ e quindi
        \[
            \pi_1\left(X_i, x_i\right) \cong \Z
        \]
        Quindi se $x_0 \in X_1 \cap X_2$, usando la versione debole del teorema di Seifert-van Kampen si ha che
        \[
            \pi_1\left(\Disc{2} \setminus \set{x_1, x_2}, x_0\right) = \pi_1\left(X_1, x_0\right) * \pi_1\left(X_2, x_0\right) = \Z * \Z = F_2
        \]
        \item Calcolo del gruppo fondamentale del bouquet di due cerchio: \nl
        \begin{definition} [Bouquet di due spazi topologici] \nl
            Siano $X, Y$ due spazi topologici, $x_0 \in X, y_0 \in Y$ due punti. \nl
            Si definisce il \textbf{bouquet di $\left(X, x_0)\right)$ e $\left(Y, y_0)\right)$} lo spazio topologico:
            \[  
                \left(X, x_0)\right) \vee \left(Y, y_0)\right) := \left(X \sqcup Y\right) / \set{x_0, y_0}
            \]
        \end{definition}

        Sia quindi $X = \Sph{1} \vee \Sph{1}$ il bouquet di due cerchi. \nl
        Dato un punto $x_1$ nel primo cerchio ed un punto $x_2$ nel secondo cerchio, si scrive $X = X_1 \cup X_2$
        dove $X_1 = X \setminus {x_2}$ e $X_2 = X \setminus {x_1}$ sono due aperti di $X$.
        Si ha che $X_1 \cap X_2$ \`e semplicemente connesso, quindi si pu\`o applicare la versione debole del teorema di Seifert-van Kampen.
        \[
            \pi_1\left(X, x_0\right) = \pi_1\left(X_1, x_0\right) * \pi_1\left(X_2, x_0\right) = \Z * \Z = F_2
        \]
        \item Iterando gli argomenti sopracitati si pu\`o dimostrare che il gruppo fondamentale del disco senza $n$ punti \`e
        $F_n$ e il gruppo fondamentale del bouquet di $n$ cerchi \`e $F_n$.
\end{enumerate}

\begin{definition} [Grafo finito] \nl
    Uno spazio di \textbf{Hausdorff} $X$ si dice grafo finito se: \nl
    $\exists X_0 \subset X$ sottospazio finito e discreto tale che $X \setminus X_0$ \`e
    unione disgiunta di un numero finito di aperti $e_1, e_2, \ldots, e_n$ tali che: \nl
    Ogni $e_i$ \`e omeomorfo a un intervallo aperto $(0, 1)$ e $\abs{\bar{e_i} \setminus e_i} \leq 2$ \nl
    Inoltre deve valere che: \nl
    se $\abs{\bar{e_i} \setminus e_i} = 2$ allora
        \[ 
            \left(\bar{e_i}, e_i\right) \cong \left([0,1], (0,1)\right)
        \]
    in tale caso si dice che $e_i$ \`e un \textbf{arco} del grafo. \nl
    se $\abs{\bar{e_i} \setminus e_i} = 1$ allora
        \[
            \left(\bar{e_i}, e_i\right) \cong \left(\Sph{1}, \Sph{1} \setminus \set{pt.}\right)
        \]
    In tal caso si dice che $e_i$ \`e un \textbf{ciclo} del grafo. \nl
    L'insieme $X_0$ \`e detto \textbf{insieme dei vertici} del grafo. \nl
    Un grafo si dice \textbf{albero} se \`e conesso e non contiene cicli.
\end{definition}

\begin{theorem} [Gruppo fondamentale di un grafo finito] \nl
    Sia $X$ un grafo finito, allora il gruppo fondamentale $\pi_1(X, x_0)$ \`e un gruppo libero e finitamente generato
    \[
        \pi_1(X, x_0) \cong F_n
    \]
    dove $n$ \`e il numero di cicli del grafo.
\end{theorem}

\begin{lemma} \nl
    Gli alberi sono contraibili ad un punto, quindi sono semplicemente connessi.
\end{lemma}

\begin{proof} \nl
    Se $X$ \`e un albero, esiste un vertice $x_0 \in X_0$ tale che \`e connesso ad un solo altro vertice,
    altrimenti $X$ conterrebbe un ciclo, sia $e_0$ un tale arco. \nl
    $X$ si retrae per deformazione su $X \setminus \left(e_0 \cup \set{x_0}\right)$, che \`e un grafo finito con un vertice in meno. \nl
    Per induzione su $n = \abs{X_0}$, $X$ \`e contraibile ad un punto.
\end{proof}

\begin{lemma} [Esistenza dello "spanning tree" di un grafo]
    Ogni grafo finito $X$ contiene un sottografo $Y \subset X$ che \`e un albero
    e ha gli stessi vertici di $X$, cio\`e $Y_0 = X_0$.
\end{lemma}

\begin{proof} \nl
    Per induzione su $n = \abs{X_0}$, se $n = 1$ allora $Y = X$ \`e un albero. \nl
    Per il passo passo si scegla un vertice a caso $x_0 \in X_0$ e si considera il grafo
    $Z$ ottenuto da $X$ rimuovendo $x_0$ ed ogni arco che lo contenga. \nl
    Le componenti connesse di $Z$ hanno numero di vertici strettamente inferiore ad $n$ e quindi
    per ipotesi induttiva ammettono uno spanning tree. \nl
    Lo spanning tree di $X$ si ottiene riunendo gli spanning tree delle componenti connesse di $Z$ aggiungendo $x_0$ e gli archi rimossi in precedenza.
\end{proof}

\begin{proof} [del teorema] \nl
    Dato $X$ un grafo, consideriamo il suo spanning tree $Y$ che esiste per il secondo lemma.
    Per il secondo lemma, lo spazio ottenuto contraendo $Y$ ad un punto $y_0 \in Y_0$ \`e un bouquet
    di $n$ cerchi, dove $n$ \`e il numero di cicli di $X$. \nl
    Si ha quindi che il gruppo fondamentale di $X$ \`e isomorfo al gruppo fondamentale del bouquet di $n$ cerchi, che \`e il gruppo libero su $n$ generatori.
\end{proof}

\subsection{Gruppi fondamentali di superfici topologiche compatte}

\begin{proposition} [Gruppo fondamentale del prodotto di spazi] \nl

\end{proposition}

\begin{proposition} [Gruppo fondamentale del toro] \nl
    Il gruppo fondamentale del toro \`e isomorfo al prodotto diretto di due gruppi ciclici:
    \[
        \pi_1\left(\Torus{2}, x_0\right) \cong \Z \times \Z
    \]
\end{proposition}

\begin{proof} \nl
    La dimostrazione seguirebbe in maniera ovvia dalla proposizione precedente, per\`o se ne da una dimostrazione
    alternativa pi\`u difficile ma pi\`u interessante. \nl
    Si considera il quadrato $Q = [0,1] \times [0,1]$ e si vede un toro come il quoziente di $Q$ per la relazione di equivalenza
    che identifica di tutti i lati opposti. \nl
    Siano $y \in Q$ il centro del quadrato, $U = Q \setminus \set{y}$ e infine data la proiezione al quoziente
    $\pi: Q \to \Torus{2}$, si consideri $V = \pi\left(\interior{Q}\right)$. \nl
    Dall'identificazione dei lati opposti segue che $V$ \`e omeomorfo a $\Torus{2} \setminus (\Sph{1} \vee \Sph{1})$. \nl
    Dato $x_0 \in U$ e $x_1 \in U \cap V$, si possono ora calcolare $\pi_1(U, x_0), \pi_1(V, x_1)$ e $\pi_1(U \cap V, x_1)$: \nl
    \begin{itemize}
        \item Il quadrato senza il centro $U = Q \setminus \set{y}$ si retrae per deformazione sul bordo $\boundary{Q}$. \nl
res        Le retrazioni per deformazione passano al quoziente e quindi $U$ \`e omotopicamente equivalente a $\pi(\boundary{Q}) \cong \Sph{1} \vee \Sph{1}$. \nl
        \item $V$ \`e l'immagine tramite la proiezione al quoziente di $\interior{Q}$, che \`e semplicemente connessa, dunque \`e semplicemente connesso. \nl
        \item $U \cap V$ \'e omeomorfo a $\Disc{2} \setminus \set{0}$ e dunque $\pi_1(U \cap V, x_1) \cong \Z$.
    \end{itemize}
    A questo punto si pu\`o applicare il teorema di Seifert-van Kampen nel caso in cui uno dei due fattori \`e banale,
    dato che $\pi_1(V) \cong {1}$, e quindi si ha che
    \[
        \pi_1\left(\Torus{2}, x_1\right) = \pi_1\left(U,x_1\right) / N,
    \]
    dove $N$ \`e il sottogruppo normale generato dall'immagine $i_*(\pi_1(U \cap V))$ dove 
    $i: U \cap V \into U$ \`e l'inclusione naturale. \nl
    Il gruppo fondamentale $\pi_1(U, x_0)$ \`e generato da due generatori $a, b$ che corrispondono alle classi di omotopia dei lacci che girano intorno ai cerchi. \nl
    Quindi se $f: I \to U$ \`e il cammino che collega $x_0$ ad $x_1$ vale che il gruppo fondamentale $\pi_1(U, x_1)$ \`e generato
    da $\tau_f(a)$ e $\tau_f(b)$. \nl
    D'altra parte $\pi_1(U \cap V, x_1)$ \`e generato da un laccio $c$ che gira intorno ad $y$ il centro del quadrato. \nl
    Deformando $c$ sul bordo e passando poi al quoziente, si ha che
    \[
        c \sim \tau_f(a*b*a^{-1}*b^{-1}).
    \]
    Si ha quindi che l'immagine $i_*(\pi_1(U \cap V))$ \`e generata da $a*b*a^{-1}*b^{-1}$, si conclude quindi Che
    \[
        \pi_1(\Torus{2}, x_1) = \left< a, b \mid [a,b]\right> \cong \Z \times \Z,
    \]
    dove $[a,b] = a*b*a^{-1}*b^{-1}$ \`e il commutatore.
\end{proof}

\begin{proposition} [Gruppo fondamentale del toro con due buchi] \nl
    Si indicher\`a con $\gTorus{2}$ la superficie simile al toro, per\`o con 2 buchi, cio\`e una doppia ciambella.
    \[
        \pi_1(\gTorus{2}) = \left< a_1, b_1, a_2, b_2 \mid [a_1, b_1][a_2, b_2]\right>
    \]
\end{proposition}

\begin{proof} \nl
    Il toro con due buchi $T = \gTorus{2}$, si pu\`o identificare con la somma connessa
    \[
        T = T_1 \# T_2,
    \]
    dove $T_1, T_2$ sono due tori $\Torus{2}$, ottenuta incollando i due tori su dei dischi $\Disc{2}$ presi su ciascuno dei due tori. \nl
    Come si pu\`o ottenere questa costruzione come quoziente? \nl
    Si prendono due copie di $Q$. Si identificano tra di loro i lati opposti ai vertici di ciascun quadrato e
    per identificare i dischi detti prima, si considerano due lacci $c_1, c_2$ che partono da uno dei vertici di ciascun quadrato e si identificano tra loro. \nl
    In sostanza i due quadrati diventano due pentagoni, con due coppie di lati opposti identificate tra loro, ed con il lato $c_1$ di uno identificato con il lato $c_2$ dell'altro. \nl
    Perci\`o incollando $c_1$ e $c_2$ si ottiene un ottagono $O$ i cui lati alterni sono identificati con direzione opposta. \nl
    Conoscendo la costruzione di $T$ come quoziente, si pu\`o calcolare il suo gruppo fondamentale come nel caso di $\Torus{2}$. \nl
    Si consideri $U = T \setminus \set{y}$ e $V = i_*(\interior{O})$ \nl
    $U$ \`e omotopicamente equivalente al bouqet di 4 cerchi dunque $\pi_1(U) \cong F_4$, mentre $V$ \`e semplicemente connesso.
    $U \cap V \equiv \Disc{2} \setminus {0}$ e dunque $\pi_1(U \cap V) \cong \Z$ ed il generatore \`e
    \[
        [a_1, b_1][a_2, b_2]
    \]
    Quindi si conclude la tesi usando Van Kampen nel caso in cui uno dei fattori \`e semplicemente connesso.
\end{proof}

\begin{proposition} [Gruppo fondamentale di un $g$-Toro] \nl
    Un toro con $g$ si pu\`o vedere come somma connessa $T_1 \# T_2 \# \dots T_g$ di $g$ tori, quindi in maniera analoga al caso precedente
    si pu\`o vedere come quozietne di un ($2g$-gon) $O$ con la giusta identificazione. \nl
    Quindi il gruppo fondamentale di un toro con $g$ buchi \`e dato da:
    \[
        \pi_1(\gTorus{g}) = \left< a_1, b_1, a_2, b_2, \ldots, a_g, b_g \mid [a_1, b_1][a_2, b_2] \ldots [a_g, b_g]\right>
    \]
\end{proposition}

\begin{proposition} [Gruppo fondamentale del piano proiettivo reale] \nl
    Il gruppo fondamentale del piano proiettivo reale \`e isomorfo al gruppo ciclico di ordine 2:
    \[
        \pi_1(\Proj{\R}{2}, x_0) = \left< a \mid a^2 \right> \cong \Z / 2\Z
    \]
    e inoltre la somma connessa di $g$ piani proiettivi ha come gruppo fondamentale:
    \[
        \pi_1(\Proj{\R}{2} \# \Proj{\R}{2} \# \dots \Proj{\R}{2}) = \left< a_1, a_2, \dots a_g \mid a_1^2a_2^2 \dots a_g^2\right>
    \]
\end{proposition}

\begin{proof}
    Da fare per esercizio
\end{proof}

\begin{definition} [Superficie topologica] \nl
    Una superficie topologica \`e uno spazio di Hausdorff connesso $X$ che \`e compatto che ammette un ricoprimento
    $\set{U_i}_{i \in I}$ di aperti $U_i$ tali che $U_i \cong D^2 \quad \forall i$.
\end{definition}

\begin{theorem} (Classificazione delle superfici topologiche) \nl
    Le classi di omeomorfismo delle superfici topologiche sono date da:
    \begin{enumerate}
        \item La sfera $\Sph{2}$.
        \item I $g$-tori $\gTorus{g}$, cio\`e le superfici ottenute come somma connessa di tori.
        \item Le somme connesse di $g$ piani proiettivi reali $\Proj{\R}{2} \# \Proj{\R}{2} \# \dots \Proj{\R}{2}$.
    \end{enumerate}
    
\end{theorem}

\section{Rivestimenti}
\subsection{Definizioni ed esempi}

\begin{remark} [Le componenti connesse di uno spazio localmente connesso]
    Se $X$ \`e uno spazio topologico localmente connesso, allora le sue componenti connesse sono aperte e chiuse.
\end{remark}

\begin{proof}
    Da scrivere, dovrebbe essere facile. \nl
\end{proof}

Tutti gli spazi topologici saranno supposti localmente connessi per archi, e dunque localmente connessi. \nl
Dunque per l'osservazione precedente le componenti connesse saranno sempre aperte.

\begin{definition} [Rivestimento] \nl
    Dato uno spazio topologico $X$, un rivestimento per $X$ \`e una coppia $\left(Y, p\right)$, dove
    $Y$ \`e uno spazio topologico e $p: Y \to X$ \`e una mappa continua tale che: \nl
    $\forall x \in X \quad \exists U_x$ intorno di $x$ aperto detto \textbf{intorno ben rivestito di $x$} tale che 
    \[p^{-1}(U_x) = \bigsqcup_{i \in I} U_i,\]
    e $p_{\big| U_i}: U_i \to U_x$ \`e un omeomorfismo $\forall i \in I$. \nl
\end{definition}

\begin{remark}
    Un rivestimento \`e sempre un'applicazione aperta e omeomorfismo locale.
\end{remark}

\begin{example} [Rivestimento banale] \nl
    Sia $X$ uno spazio topologico e $F$ uno spazio topologico discreto, allora $Y = X \times F \cong \bigsqcup_{f \in F} X$ e la mappa $p: Y \to X$ di proiezione sulla prima coordinata
    \`e un rivestimento di $X$ detto rivestimento banale.
\end{example}

\begin{example} [Nastro di Moebius] \nl
    Sia $M$ il nastro di Moebius, e $\boundary{M}$ il suo bordo, la proiezione $p: \boundary{M} \to \Sph{1}$ data dalla proiezione
    del bordo sull $\Sph{1}$ centrale del nastro \`e un rivestimento di $\Sph{1}$. \nl
    Un tale rivestimento non \`e banale perch\'e $M \not\cong \Sph{1} \sqcup \Sph{1}$.
\end{example}

\begin{example} [Mappa esponenziale] \nl
    La mappa $p: \R \to \Sph{1} \subset \C: t \mapsto e^{2 \pi i t}$ \`e un rivestimento.
\end{example}

\begin{example} [Rivestimento di $\Sph{n}$ in se stesso] \nl
    Se $n \neq 1$ la mappa $p: \Sph{n} \to \Sph{n}: e^{2 \pi \theta} \mapsto e^{2 \pi n \theta}$ \`e un rivesitmento
    di $\Sph{n}$ dove $\forall x \in \Sph{n}$ si ha che $\abs{p^{-1}(x)} = n$. 
\end{example}

\begin{example} [Rivestimento di $\C \setminus \set{0}$] \nl
    La mappa $p: \C \setminus \set{0} \to \C \setminus \set{0}: z \mapsto z^n$ \`e un rivestimento di $\C \setminus \set{0}$.
    Mentre la mappa $\C \to \C: z \mapsto z^n$ non lo \`e.
\end{example}

\begin{proposition} [Operazioni sui rivestimenti] \nl
    \begin{enumerate}
        \item Sia $p: Y \to X$ un rivestimento e $U \subset X$ un aperto. \nl
        La mappa $\restrict{p}{p^{-1}(U)}: p^{-1}(U) \to U$ \`e un rivestimento di $U$.
        \item Se $X$ \`e connesso e $Z \subset Y$ \`e una qualunque componente connessa di $Y$, allora 
        la mappa $p_{\big| Z}: Z \to X$ \`e un rivestimento di $X$. \nl
        Infatti, se $U \subset X$ \`e un intorno ben ben rivestito per $p$, allora vale che
        \[
            p^{-1}(U_x) = \bigsqcup_{i \in I} U_i,
        \]
        per connessione di $Z$ vale per\`o che $U_i \subset Z$ oppure $U_i \cap Z = \empty.$ \nl
        Inoltre $p(Z) = X$, sia infatti $x' \not\in p(Z)$, dato $U'$ l'intorno ben rivestito di $x'$ e
        gli $U'_i$ tali che $p^{-1}(U_i) = \bigsqcup_{i \in I} U'_i$, dato che che $U_i \not\subset Z$ vale che
        $U_i \cap Z = \empty $ per ogni $i$. (da controllare)
        \item Siano $p_1: Y_1 \to X_1, p_2: Y_2 \to X_2$ due rivestimenti rispettivamente di $X_1$ e $X_2$. Allora la mappa
        \[  
            \left(p_1, p_2\right): Y_1 \times Y_2 \to X_1 \times X_2: (y_1, y_2) \to (p_1(y_1), p_2(y_2))
        \]
        \`e un rivestimento di $X_1 \times X_2$. \nl
        Ad esempio $\R^2 \to T = \Sph{1} \times \Sph{1}$ \`e un rivestimento del toro.

    \end{enumerate}
\end{proposition}

\subsection{Morfismi di rivestimenti}

\begin{definition} [Morfismo di rivestimenti] \nl
    Un \textbf{morfismo di rivestimenti} \`e il dato di due rivestimenti $pi_1: Y_1 \to X, p_2: Y_2 \to X$
    ed una mappa continua $\varphi: Y_1 \to Y_2$ tale che
    \[
        p_2 \circ \varphi = p_1
    \]
    ovvero il seguente diagramma commuta:
    \[\begin{tikzcd}
	{Y_1} && {Y_2} \\
	& X
	\arrow["\varphi", from=1-1, to=1-3]
	\arrow["{p_1}", from=1-1, to=2-2]
	\arrow["{p_2}"', from=1-3, to=2-2]
    \end{tikzcd}\]
\end{definition}

\begin{definition} [Isomorfismo di rivestimenti] \nl
    Se $\varphi: Y_1 \to Y_2$ \`e un morifmso di rivestimenti come sopra tale che esiste un altro morfismo
    di rivestimenti $\psi: Y_2 \to Y_1$ tale che $\psi \circ \varphi = id_{Y_1}$ e $\varphi \circ \psi = id_{Y_2}$
    si dice che $\varphi$ \`e un isomorifsmo tra i rivestimenti $pi_1: Y_1 \to X, p_2: Y_2 \to X$.
\end{definition}

\begin{definition} [Morfismi tra mappe continue] \nl
    La stesse definizioni date sopra si possono dare nel caso generale in cui $p_1, p_2$ sono generiche mappe continue.
\end{definition}

\begin{proposition} [Caratterizzazione dei rivestimenti] \nl
    Sia $p: Y \to X$ una mappa continua surgettiva. \nl
    \[
        p \text{ \`e un rivestimento} \iff \forall x \in X \quad \exists U \tc p_{|p^{-1}(U)}: p^{-1}(U) \to U \text{ \`e isomorfo ad un rivestimento banale}
    \]
\end{proposition}

\begin{proof} \nl
    $(\Leftarrow)$ Dovrebbe essere facile, ma \`e da scrivere. \nl
    $(\Rightarrow)$ Se $p^{-1}(U) = \bigsqcup_{i \in I} U_i$, si pu\`o munire $I$ della topologia discreta
    e in questo modo $p^{-1}(U) \cong I \times U$ e la mappa $\varphi: Y \to I \times U: y \mapsto (i, p(y))$ se $y \in U_i$ \`e un omeomorfismo e dunque un isomorifsmo tra i due rivestimenti.
\end{proof}

\begin{corollary} [Fibre di un rivestimento] \nl
    Se $X$ \`e uno spazio topologico connesso e $p: Y \to X$ \`e un rivestimento, allora tutte le fibre hanno la stessa cardinalit\`a. \nl
    Cio\`e \[
        \forall x,y \in X \quad \abs{p^{-1}(x)} = \abs{p^{-1}(y)}.
    \]
\end{corollary}

\begin{definition} [Grado di un rivestimento] \nl
    Il corollario precedente permette di definire il \textbf{grado} di un rivestimento $p: Y \to X$ come la cardinalit\`a di una qualunque delle fibre di $p$.
\end{definition}

\begin{proof}
    Mostriamo che per ogni $\alpha$ "cardinale" l'insieme $X_{\alpha} = \set{x \in X \mid \abs{p^{-1}(x)} = \alpha}$ \`e sia aperto che chiuso. \nl

    Da finire.
\end{proof}

\subsection{Azioni propriamente discontinue}

\begin{definition} [Azione di gruppo propriamente discontinua] \nl
    Si dice che un gruppo $G$ agisce in maniera propriamente discontinua su uno spazio topologico $Y$ se
    ogni punto $y \in Y$ ammette un intorno $U_y$ tale che $\forall g,h \in G \quad g.U \cap h.U = \emptyset$ se $g \neq h$.
\end{definition}

\begin{proposition} [Rivestimenti dati da azioni propriamente discontinue] \nl
    Sia $Y$ uno spazio topologico e $G$ un gruppo che agisce in maniera propriamente discontinua su $Y$.
    Allora la mappa $p: Y \to Y/G$ data dalla proiezione di $Y$ su $Y/G$ \`e un rivestimento di $Y/G$.
    Inoltre gli intorni ben rivestiti di ogni punto $x \in Y/G$ sono dati dall'immagine degli intorni che rendono l'azione propriamente discontinua.
\end{proposition}

\begin{proof} \nl
    Chiaramente la proiezione al quoziente \`e sempre surgettiva ed \`e continua per definizione della topologia quoziente. \nl
    Per ogni punto $x \in Y$ sia $U$ l'intorno che rende propriamente discontinua l'azione, allora vale
    \[
        p^{-1}(p_g(U)) = \bigsqcup_{g \in G} g.U,
    \]
\end{proof}

\begin{example} \nl
    Si consider l'azione propriamente discontinua di $\Z$ su $\R$ data da $n . x = x + n$, allora la mappa
    \[
        p: \R \to \R / \Z \cong \Sph{1}: x \mapsto x + \Z
    \]
    \`e un rivestimento che possiamo identificare come il rivestimento $t \to e^{2 n \pi i t}$ di $\Sph{1}$ visto in precedenza.
\end{example}

\begin{example} \nl
    Il rivestimento $\R^2 \to T = \Sph{1} \times \Sph{1}$ pu\`o essere visto come il rivestimento indotto
    dall'azione propriamente discontinua di $\Z^2$ su $\R^2$ data da $(m, n) . (x, y) = (x + m, y + n)$.
\end{example}

\begin{example} \nl
    Anche il rivestimento $\C \setminus \set{0} \to \C \setminus \set{0}$ visto in precedenza pu\`o essere visto come il rivestimento indotto dall'azione propriamente discontinua
     del gruppo ciclico di $n$ elementi $\set{\zeta_n \in \C \mid \zeta_n^n =1}$ su $\C \setminus \set{0}$ data da
     \[
        \zeta_n. z = \zeta_n z
     \]
\end{example}

\begin{example} [Spazi proiettivo reale] \nl
    Si considera l'azione propriamente discontinua di $\Z / 2\Z$ su $\Sph{n} = \set{x \in \R^n \mid \norm(x) = 1}$ data da
    \[
        \tau.x = -x \text{ se } \tau \text{ \`e l'elemento non banale di } \Z / 2\Z.
    \]
    Questa azione \`e propriamente discontinua e la proiezione al quoziente \`e un rivestimento $p: \Sph{n} \to \Proj{\R}{n}$.
\end{example}

\begin{example} [Rivestimento banale] \nl
    Sia $G$ un gruppo topologico munito della topologia discreta, allora l'azione naturale di $G$ su $G \times X$
    data da 
    \[
        g.\left(g',x\right) = \left(gg', x\right)
    \]
    e propriamente discontinua ed induce il rivestimento banale $G \times X \to X$. \nl
    Inoltre per ogni sottogruppo normale $H \normal G$, la mappa $G \times X \to G/H \times X$ \`e ancora
    un rivestimento banale.
\end{example}

\begin{definition} [Automorfismo di rivestimenti] \nl
    Un automorfismo del rivestimento $p: Y \to X$ \`e un isomorfismo di $p$ con se stesso, cio\`e
    \[\begin{tikzcd}
	{Y} && {Y} \\
	& X
	\arrow["\varphi", from=1-1, to=1-3]
	\arrow["{p}", from=1-1, to=2-2]
	\arrow["{p}"', from=1-3, to=2-2]
    \end{tikzcd}\]
    \`E facile verificare che l'insieme $\Aut{Y/X} := \set{\phi: Y \to Y \mid \phi \text{ automorfismi di p in se stesso}}$
    forma un gruppo con l'operazione di composizione ed \`e detto \textbf{gruppo di automorfismi del rivestimento $Y/X$}.
\end{definition}

\begin{remark} \nl
    Nel caso del rivestimento $Y \to Y/G$ dato dalla proiezione, la mappa 
    \[
        G \to \Aut{X/G}: g \mapsto \varphi_g(x \mapsto g.x)
    \]
    \`e un omomorfismo iniettivo $G \into \Aut(X/G)$.
\end{remark}

\begin{proposition} 
    Se $Y$ \`e connesso allora la mappa, come sopra
    \[
        G \to \Aut{X/G}: g \mapsto \varphi_g(x \mapsto g.x)
    \]
    \`e un isomorfismo tra $G$ e $Aut(X/G)$.
\end{proposition}

\begin{lemma} \nl
    Sia $Y \to X$ un rivestimento connesso e $\varphi \in \Aut{Y/X}$, vale che
    \[
    \exists y \in Y \tc \varphi(y) = y \implies \phi = \id
    \]
    Cio\`e l'unico automorfismo di rivestimento che lascia fisso un punto \`e l'identit\`a. \nl
    Quindi gli automorfismi di rivestimenti indotti da azioni propriamente discontinue non hanno punti fissi.
\end{lemma}

\begin{proof} [Lemma $\implies$ Proposizione] \nl
    Grazie all'osservazione, resta da verificare che tale mappa \`e surgettiva. \nl
    Cio\`e $\forall \varphi \in \Aut{Y/X} \quad \exists g \in G$ tale che $\varphi = \varphi_g$. \nl
    siccome $\varphi$ \`e un automorfismo di rivestimenti vale che $p \circ\varphi = p$, quindi
    \[\forall y \in Y \quad p(\varphi(y)) = p(y).\]
    ma quindi $\varphi(y)$ \`e un elemento di $p^{-1}(p(y)) = \set{g.y \mid g \in G}$ essendo la fibra di un punto tramite la proiezione al quoziente. \nl
    Dunque $\exists g \in G$ tale che $\varphi(y) = g.y$, e quindi $\varphi_g \circ \varphi$ fissa il punto $y$, ma grazie al lemma si conclude che $\varphi_g \circ \varphi = id_Y$. \nl
    Quindi $\varphi = \varphi_g$ e dunque la mappa \`e surgettiva.
\end{proof}

\begin{proposition} \nl
    Siano $p: Y \to X$ un rivestimento cnnesso e $Z$ uno spazio topologico connesso. Siano inoltre 
    $f, g: Z \to X$ due mappe continue tali che $p \circ f = p \circ g$. Vale che 
    %%\[\exists z \in Z \tc f(z) = g(Z) \quad \implies \quad f = g\]
    %%\[\begin{tikzcd}
	%%Z & Y & X
	%%\arrow["g"', shift right, from=1-1, to=1-2]
	%%\arrow["f", shift left, from=1-1, to=1-2]
	%%\arrow["{p \circ g}"', curve={height=12pt}, from=1-1, to=1-3]
	%%\arrow["{p \circ f}", curve={height=-12pt}, from=1-1, to=1-3]
	%%\arrow["p"', from=1-2, to=1-3]
    %%\end{tikzcd}\]
\end{proposition}

\begin{remark} \nl
    Il lemma precedente \`e il caso particolare del teorema appena enunciato, nel caso in cui $Z = Y$, $f = \varphi$ e $g = id_Y$.
\end{remark}

\begin{proof}
    Sia $z \in Z$ tale che $f(z) = g(z)$, e sia $x = p(f(z)) = p(g(z))$, che sono uguali per l'ipotesi. \nl
    Sia $U$ un intorno ben rivestito di $x$ e siano $U_i$ gli intorni che compongono la fibra di $p^{-1}(U)$, cio\`e
    \[
        p^{-1}(U) = \bigsqcup_{i \in I} U_i.
    \]
    Poich\'e $p$ \`e una funzione e $p \circ f = p \circ g$, deve esistere un intorno $U_i$ tra quelli sopra tale che $f(z) = g(z) \in U_i$. \nl
    Si \`e mostrato che l'insieme
    \[S := \set{z \in Z \mid f(z) = g(Z)}\]
    \`e non vuoto, mostrando che $S$ e\` sia aperto che chiuso, si conclude per connessione che $S = Z$. \nl
    $S$ \`e aperto perch\'e intorno di ogni suo punto, infatti per ogni punto $z$ tale che $f(z) = g(z)$ dalla continuit\`a di $f$ e $g$ segue che
    esiste un intorno aperto $V$ di $z$ tale che $f(V), g(V) \subset U_i$, ma poich\'e $p \circ f = p \circ g$ e
    la restrizione di $p$ ad $U_i$ \`e un omeomorfismo, vale che $\forall z' \in V \quad f(z') = g(z')$ e quindi $S$ \`e intorno di $z$.
    Si dimostra ora in maniera analoga che $S' := Z \setminus S = \set{z \in Z \mid f(z) \neq g(z)}$ \`e aperto e quindi $S$ \`e anche chiuso. \nl
    Infatti, $\exists i \neq j$ tali che $f(z) \subset U_i$ e $g(z) \subset U_j$, allora per continuit\`a, come prima
    $\exists V$ intorno aperto di $z$ tale che $f(z) \subset U_i$ e $g(z) \subset U_i$ e come prima
    segue che $\forall z' \in V \quad f(z') \neq g(z')$. 
\end{proof}

\begin{proposition} [I rivestimenti sono dati da azioni propriamente discontinue] \nl
    Se $Y \to X$ \`e un rivesitmento connesso. l'azione di $\Aut(Y/X)$ su $Y$ data da
    \[ \varphi.y = \varphi(y)\]
    \`e propriamente discontinua.
\end{proposition}

\begin{proof}

\end{proof}

\subsection{Teoria di Galois per rivestimenti}

\begin{definition} [Rivestimento di Galois] \nl
    Dalla proposizione precedente se $p: Y \to X$ \`e un rivestimento connesso si ha la fattorizzazione di $p$ data da:
    \[\begin{tikzcd}
	Y & {Y/\Aut(Y/X)} & X
	\arrow["\pi", from=1-1, to=1-2]
	\arrow["p"', curve={height=12pt}, from=1-1, to=1-3]
	\arrow["{\bar{p}}", from=1-2, to=1-3]
    \end{tikzcd}\]
    Se $\bar{p}$ \`e un omeomorfismo si dice che il rivestimento \`e di Galois.
\end{definition}

\begin{remark}
    Se $Y$ \`e connesso e $G$ agisce su $Y$ in maniera propriamente discontinua, il rivestimento $Y \to Y/G$ \`e di Galois.
\end{remark}

\begin{theorem} [Teorema di Galois per rivestimenti] \nl
    Sia $Y \to X$ un rivestimento di Galois e $G := Aut{Y/X}$. Vale che \nl
    $\forall H \sub G$ sottogruppo, la mappa $Y/H \to Y$ \`e un rivestimento connesso e
    \[  
        H \normal G \quad \iff \quad Y/H \to Y \text{ \`e un rivestimento di Galois}.
    \]
    Inoltre, se $Z \to X$ \`e un rivestimento connesso tale che esiste un morfismo $\varphi$ di rivestimenti che fa commutare il seguente diagramma:
    \[\begin{tikzcd}
    Z && Y \\
    & X
    \arrow["\varphi", from=1-1, to=1-3]
    \arrow[from=1-1, to=2-2]
    \arrow[from=1-3, to=2-2]
    \end{tikzcd}\]
    si ha che $Y \to Z$ \`e un rivestimento di Galois e $\Aut{Y/Z} \sub G$  \nl
    Dunque il teorema da una corrispodenza:
    % https://q.uiver.app/#q=WzAsMixbMCwwLCJcXHNldHtIIFxcc3ViIEcgXFx0ZXh0eyBzb3R0b2dydXBwaX0gfSJdLFsyLDAsIlxcc2V0e3BfejpaIFxcdG8gWSBcXHRleHR7IHJpdmVzdGltZW50aSBjb25uZXNzaX0gXFxtaWQgcF9aIFxcY2lyYyBcXHZhcnBoaSA9IHBfWX0iXSxbMCwxLCIiLDAseyJzdHlsZSI6eyJ0YWlsIjp7Im5hbWUiOiJhcnJvd2hlYWQifX19XV0=
    \[\begin{tikzcd}
	{\set{H \sub G \text{ sottogruppi} }} && {\set{p_z:Z \to Y \text{ rivestimenti connessi} \mid p_Z \circ \varphi = p_Y}}
	\arrow[tail reversed, from=1-1, to=1-3]
\end{tikzcd}\]
\end{theorem}

\begin{example} \nl
    Il rivestimento $\R \to \R \setminus \Z$ \`e di Galois.
\end{example}

\begin{example} [Esempio di rivestimento non di Galois] \nl

\end{example}

\subsection{RIvestimento universale}

\begin{definition} [Rivestimento universale] \nl
    Un rivestimento $\pi: \tilde{X} \to X$ si dice \textbf{universale} se $\tilde{X}$ \`e semplicemente connesso.
\end{definition}

\begin{proposition} [Propriet\`a universale del rivestimento universale]
    Sia $\pi: \tilde{X} \to X$ un rivestimento universale, per ogni altro rivestimento $p: Y \to X$ esiste un morfismo
    di rivestimenti $\varphi: \tilde{X} \to X$ e fissati $x_0 \in X$, $\bar{x_0}\in \pi^{-1}(x_0)$, $y_0 \in p^{-1}(x_0)$
    ne esiste uno solo tale che $\varphi(\bar{x_0}) = y_0$
\end{proposition}

\begin{theorem} [Sollevamento di cammini e omotopie] \nl
    Sia $p: Y \to X$ un rivestimento, $x_0 \in X$ e $y \in p^{-1}(x)$. Vale \nl
    \begin{enumerate}
        \item Se $f: I \to X$ \`e un cammino tale che $f(0) = x_0$, allora
            $\exists! \tilde{f}: I \to Y$ cammino tale che $p \circ \tilde{f} = f$ e $\tilde{f}(0) = y$
            \[\begin{tikzcd}
            I & Y \\
            & X
            \arrow["{\tilde{f}}", from=1-1, to=1-2]
            \arrow["f"', from=1-1, to=2-2]
            \arrow["p", from=1-2, to=2-2]
        \end{tikzcd}\]
        \item Se $f \sim g$ sono due cammini omotopi allora $\tilde{f}(1) = \tilde{g}(1)$ e $\tilde{f} \sim \tilde{g}$ 
    \end{enumerate}
\end{theorem}

\begin{proof}
    Vedere da Frigerio, troppo pi\`u chiara.
\end{proof}

\begin{proof} [della propriet\`a universale] \nl
    
\end{proof}

\begin{corollary} \nl
    Dato $\varphi: \tilde{X_1} \to \tilde{X_2}$ morfismo tra rivestimenti universali
    \[\begin{tikzcd}
	{\tilde{X_1}} && {\tilde{X_2}} \\
	& X
	\arrow["\varphi", from=1-1, to=1-3]
	\arrow["{\pi_1}"', from=1-1, to=2-2]
	\arrow["{\pi_2}", from=1-3, to=2-2]
    \end{tikzcd}\]
    $\varphi$ \`e un isomorfismo.
\end{corollary}

\begin{corollary} [Unicit\`a del rivestimento universale] \nl
    Il rivestimento universale di un qualunque spazio topologico $X$ \`e unico a meno di isomorfismo.
\end{corollary}

\begin{corollary} \nl
    Ogni rivestimento di uno spazio $X$ semplicemente connesso \`e banale.
\end{corollary}

\begin{theorem} [Gruppo di Galois e gruppo fondamentale] \nl
    Un rivestimento universale $\tilde{X} \to X$ \`e sempre di Galois e 
    \[
        \Aut{\tilde{X}/X} \cong \pi_1(X, x_0) \quad \forall x_0 \in X
    \]
\end{theorem}

\begin{remark}
    Questo teorema permette di ricalcolare il gruppo fondamentale di alcuni spazi topologici precedenti:
    \begin{enumerate}
        \item $\R \to \Sph{1}$ \nl
        \item $\R^2 \to \Sph{1} \times \Sph{1}$ \nl
        \item $\Sph{n} \to \Proj{R}{n}$
    \end{enumerate}
\end{remark}

\begin{lemma} [Caratterizzazione dei rivestimenti di Galois] \nl
    Dato $p: Y \to X$ rivestimento connesso. \nl
    \[
        p \text{ \`e di Galois} \iff \exists x \in X \tc \Aut{Y/X} \text{ agisce transitivamente su } p^{-1}(x)
    \]
\end{lemma}

\begin{definition} [Spazio localmente semplicemente connesso] \nl
    Uno spazio topologico $X$ si dice localmente semplicemente connesso se ogni ogni punto ammette un sistemfa fondamentale
    di intorni fatto di intorni semplicemente connessi.
\end{definition}

\begin{theorem} [Esistenza del rivestimento universale] \nl
    Sia $X$ spazio connesso e localmente semplicemente connesso, allora esiste $\tilde{X} \to X$ rivestimento universale di $X$.
\end{theorem}

\begin{proof}
    2 costruzioni spastiche.
\end{proof}

\subsection{Dimostrazione del teorema di Seifert-Van Kampen}

\begin{lemma} \nl
    Siano $p_1: Y_1 \to X_1$, $p_2: Y_2 \to X_2$ due rivestimenti e $\varphi: Y_1 \to Y_2$ un morfismo di rivestimenti, allora
    \[
        \exists p: Y \to X = x_1 \cup X_2 \text{ rivestimento tale che } \restrict{p}{p^{-1}(X_i)} \cong p_i^{-1}(X_i) \to X
    \]
    Inoltre se $p_1, p_2$ sono di Galois allora anche $p$ \`e di Galois.

    % https://q.uiver.app/#q=WzAsMyxbMCwwLCJwXzFeey0xfVxcbGVmdChYXzEgXFxjYXAgWF8yXFxyaWdodCkiXSxbMiwwLCJwXzJeey0xfVxcbGVmdChYXzEgXFxjYXAgWF8yXFxyaWdodCkiXSxbMSwxLCJYXzEgXFxjYXAgWF8yIl0sWzAsMl0sWzEsMl0sWzAsMSwiXFx2YXJwaGkiXV0=
    \[\begin{tikzcd}
	{p_1^{-1}\left(X_1 \cap X_2\right)} && {p_2^{-1}\left(X_1 \cap X_2\right)} \\
	& {X_1 \cap X_2}
    \arrow["\varphi", "\cong"', from=1-1, to=1-3]
	\arrow[from=1-1, to=2-2]
	\arrow[from=1-3, to=2-2]
    \end{tikzcd}\]
\end{lemma}

\begin{definition} [$G$-rivestimento] \nl
    Un $G$-rivestimento \`e un rivestimento della forma $Y \to Y/G$ dove $G$ \`e un gruppo che agisce in
    maniera propriamente discontinua su $Y$.
\end{definition}

\begin{theorem} \nl
    La mappa $\varrho \mapsto Y_\varrho$ induce una bigezione:
    % https://q.uiver.app/#q=WzAsMixbMCwwLCJcXHNldHtcXHZhcnJobzogXFxwaV8xKFgsIHhfMCkgXFx0byBHfSJdLFsyLDAsIlxcc2V0e1xcdGV4dHtHLXJpdmVzdGltZW50aSB9IHA6IFkgXFx0byBYIFxcbWlkIHkgPSBwXnstMX0oeF8wKSAgXFx0ZXh0eyBmaXNzYXRvfX0iXSxbMCwxLCIiLDAseyJzdHlsZSI6eyJ0YWlsIjp7Im5hbWUiOiJhcnJvd2hlYWQifX19XV0=
    \[\begin{tikzcd}
        {\set{\varrho: \pi_1(X, x_0) \to G}} && {\set{\text{G-rivestimenti } p: Y \to X \mid y = p^{-1}(x_0)  \text{ fissato}}}
        \arrow[tail reversed, from=1-1, to=1-3]
    \end{tikzcd}\]
    con i rivestimenti a meno di isomorfismo.
\end{theorem}

\clearpage
\section{Domande orali del 3 Giugno}

\subsection{Orale 1}
\begin{enumerate}
    \item Come si dimostra che il prodotto di 2 compatti \`e compatto? 
    \item Perch\'e i compatti di $\R$ sono chiusi?
           \begin{answer}
            \`E T2
           \end{answer} 
    \item Esempio di compatto di $\R$ che non \`e unione finita di intervalli chiusi e limitati.  
        \begin{answer}
            L'insieme di Cantor
        \end{answer}
    \item Se $X$ e $Y$ sono compatti e discreti, cosa si pu\`o dire? (TAMAS)
        \begin{answer}
            $X, Y$ sono finiti, quindi il prodotto di finiti \`e finito hahaha.
        \end{answer}
    \item Esempio di un compatto non discreto
        \begin{answer}
            Un intervallo chiuso.
        \end{answer}
\end{enumerate}

\subsection{Orale 2}
\begin{enumerate}
    \item Che relazione ci sono tra chiuso e compatto
        \begin{answer}
            Chiuso in un compatto \`e compatto.
        \end{answer}
    \item Quand'\`e che i sottoinsiemi compatti di $X$ sono chiusi? E se vuole mi dia un controesempio
        di un compatto non chiuso.
        \begin{answer}
            Se $X$ \`e T2 i compatti sono chiusi, il controesempio \`e un qualunque spazio con la topologia indiscreta.
        \end{answer}
    \item Sia $p: Y \to X$ un rivestimento e $U \subset X$ aperto, come si definisce la restrizione
        del rivestimento $p$ ad $U$? Dimostrare che tale restizione \`e ancora un rivestimento. (TAMAS)
        \begin{answer}
            $\restrict{p}{p^{-1}(U)}: p^{-1}(U) \to U$
        \end{answer}
    \item $X,Y$ e $U$ connessi come sopra. Quando $p^{-1}(U)$ \`e sconnesso? Dopo un po' hanno chiesto, qual \`e il rivestimento
        connesso pi\`u semplice che uno possa pensare? (TAMAS)
        \begin{answer}
            Alla prima domanda non si \`e saputo rispondere. \nl
            Alla seconda domanda $Y = X$ con $p = id_X$ che \`e un caso di omeomorfismo.
        \end{answer}
\end{enumerate}

\subsection{Orale 3}
\begin{enumerate}
    \item Com'\`e definito $\Proj{\C}{1}$? Munito della topologia quoziente a quale spazio \`e omeomorfo?
        \begin{answer}
            Alla sfera $\Sph{1}$, la dimostrazione si pu\`o fare con l'unicit\`a della compattificazione di Alexandroff perch\'e i due spazi sono $T2$ e compatti.
        \end{answer} 
    \item Chi \`e il piano all'infinito di $\Proj{\C}{1}$
        \begin{answer}
            Il punto proiettivo $[0,1]$
        \end{answer}
    \item Esercizio: Verifica che le carte affini sono omeomorfismi.
    \item Cos'\`e una conica in $\Proj{\C}{1}$? Quante coniche non degeneri esistono in $\Proj{\C}{1}$? E quante in $\Proj{\R}{1}$?
    \item Come si trova la tangente ad una conica passante per un punto?
\end{enumerate}

\subsection{Orale 4}
\begin{enumerate}
    \item Data una funzione $f: \C \to \C$ olomorfa tale che $\abs{f(z)} \leq c_1\abs{z} + c_2$ con $c_1,c_2 \in \R$ costanti.
        Cerchi una caratterizzazione di $f$.
        \begin{answer}
            I polinomi di grafo minore o uguale a $d$. Dimostrazione molto simile al Teorema di Louville.
        \end{answer}
    \item \`E vero che $f$ come sopra(quindi un polinomio) definisce un rivestimento $\C \to \C$? Per esempio $z \mapsto z^n$ \`e un rivestimento da $\C \to \C$? (Tamas)
        \begin{answer}
            Non pu\`o essere un rivestimento perch\'e $f^{-1}(0)$ per il toerema fondamentale di gruppo. \nl
            Tuttavia la mappa $\varrho: \C \setminus \set{0} \to \C \setminus \set{0}: z \mapsto z^n$ da un rivestimento
            indotto dall'azione propriamente discontinua di $\Z / n\Z$ su $\C^*$ data da $[m].(\rho e^{\theta i}) := \rho e^{\left(\theta+ m\frac{2 \pi}{n}\right)i}$. nl
            Oppure come ha risposto lui: $[m].z \to z \zeta_n^m$ dove $\zeta_n \in \C$ \`e una radice $n$ primitiva dell'unit\`a.
        \end{answer}
\end{enumerate}

\subsection{Orale 5}
\begin{enumerate}
    \item Esempio di un rivestimento di Galois e di un rivestimento non di Galois. Un esempio generale di rivestimento di Galois. (Tamas)
    \begin{answer}
        $\R \to \R / \Z$ \`e di Galois ed in generale $G$ che agisce p.d. su $\tilde{X}$ semplicemente connesso da un rivestimento $\tilde{X} \to \tilde{X} / G$ di Galois. \nl
        L'esempio non di Galois \`e quello visto a lezione.
    \end{answer}
    \item Qual \`e il gruppo fondamentale del bouquet $\Sph{1} \vee \Sph{1}$, perch\'e si poteva prevedere teoricamente? (Tamas)
    \begin{answer}
        Per il toerema di Corrispondenza di Galois esiste perch\'e ci sono sottogruppi non normali in $\Z * \Z$.
    \end{answer}
    \item Qual \`e la forma normale di una funzione $f$ memorofa con zero $z_0$ di ordine $k$? Calcoli il residuo di $h(z) = \frac{f'}{f}$ in $z_0$
    \begin{answer}
        $f(z) = (z - z_0)^k g(z)$, con $g(z) \neq 0$ olomorfa. \nl
        Se $z_0$ \`e uno zero di ordine $k$ di $f(z)$ allora \`e di ordine $k-1$ per $f'(z)$,
        quindi la funzione $h(z) = \frac{f'(z)}{f(z)}$ ha un polo di ordine $1$ in $z_0$, quindi
        \[
            Res(h, z_0) = \lim_{z \to z_0} (z - z_0)\frac{f'(z)}{f(z)} = \dots = k.
        \]
    \end{answer}
    \item Secondo lei a cosa serve? 
    \begin{answer}
        Se si integra $\frac{f'(z)}{f(z)} dz$ lungo un cammino chiuso di indice di avvolgimento $1$ intorno a $z_0$
        si ottiene $2 k \pi i$.
    \end{answer}
\end{enumerate}

\subsection{Orale 6}
\begin{enumerate}
    \item Sia $I$ un segmento che collega il polo sud ed il polo nord di $\Sph{2}$. \nl
    Si calcoli il gruppo fondamentale di $X = \Sph{2} \cup I$.
    \begin{answer}
        Si prende $X_1 = X \setminus \text{segmento chiuso proprio di I}$ e $X_2 = I \cup \text{i punti a distanza } < \varepsilon \text{ da un meridiano}$
        l'intersezione \`e semplicemente connessa, $X_2$ anche e quindi il gruppo fondamentale \`e lo stesso di $X_1$ cio\`e $\Z$
    \end{answer}
    \item Riesce a disegnare il rivestimento universale di $\Sph{1} \vee \Sph{2}$? Di che grado \`e? E quello di $X$?.
    \begin{answer}
        Il primo \`e una retta su cui si inseriscono dele sfere su ogni numero intero. \nl
        Mentre il secondo \`e dato da una "catena" di sfere collegate da dei segmenti.
    \end{answer}
\end{enumerate}

\subsection{Orale 7}
\begin{enumerate}
    \item Il prodotto munito della topologia prodotto di spazi discreti \`e discreto?
    \begin{answer}
        S\`i, se il prodotto \`e finito se \`e infinito non necessariamente altrimenti.
    \end{answer}
    \item Esempio di prodotto infinito di discreti non discreto
    \begin{answer}
        $\set{0,1}^{\N}$ che non \`e discreto perch\'e $\set{(0,0,\dots,0)}$ non \`e aperto.
    \end{answer}
    \item Prodotto numearbile di metrizzabili \`e metrizzabile? ${0,1}^{\N}$ \`e I-numerabile?
    \begin{proof}
        Si ed inoltre metrizzabile implica I-numerabile.
    \end{proof}
    \item Se $X,Y$ sono metrizzabili e compatti, dimostri che $X \times Y$ \`e compatto.
    \begin{proof}
        Usando la caratterizzazione metrizzabili: compatto se e solo se completo e totalmente limitato
    \end{proof}
    \item Cos'\`e il numero di Lebesgue?
\end{enumerate}


\subsection{Orale 8}

\subsection{Orale 9}
\begin{enumerate}
    \item Quali sono i legami tra la connession e la connessione per archi?
    \begin{answer}
        Connesso per archi implica connesso, ma non vale il viceversa.
        Per la dimostrazione si \`e dato per buono che $I$ sia connnesso.
    \end{answer}
    \item Un esempio di spazio connesso ma non connesso per archi.
    \begin{answer}
        Il seno del topologo. La dimostrazione utilizza il fatto che se $Y$ \`e connesso
        e $Y \subset Z \subset \closure{Y}$ allora $Z$ \`e connesso, ma questo fatto \`e dato per buono.
    \end{answer}
    \item Dimostra che il seno del topologo \`e connesso, senza utilizzare il lemma.
    \begin{answer}
        Se esistessero $A,B$ aperti non vuoti tali che $A \cup B = X$ e $A \cap B = \empty$. Supponendo senza perdit\`a
        di generalit\`a che $(0,0) \in A$, l'insieme $A \setminus \set{(0,0)} \neq \emptyset$(dato che $\set{0,0} \in \closure{\Gamma(sin(\frac{1}{x}))} $) \`e aperto nel grafico di $sin\left(1/x\right)$
        e quindi gli aperti $A \setminus \set{(0,0)}$ e $B$ sconnettono $sin\left(\frac{1}{x}\right)$ che per\`o \`e connesso.
    \end{answer} 
    \item Cosa sono le componenti connesse di uno spazio topologico? Se lo spazio ha una propriet\`a topologica "simpatica" cosa si sa dire?(TAMAS)
    \begin{answer}
        Se lo spazio \`e localmente connesso le componenti connesse sono aperte e chiuse.
    \end{answer}
    \item Esempio di spazi localmente connessi e di spazio non localmente connessi. Per esempio in $\R^n$.
    \begin{answer}
        Gli aperti di $\R^n$ sono sempre localmente connessi, mentre i chiusi in generale no per esempio $\set{\frac{1}{n} \mid n \in \N} \subset \R$ non \`e localmente connesso.
    \end{answer}
\end{enumerate}

    


\end{document}
