\documentclass[]{article}

% Packages
\usepackage[utf8]{inputenc} % For UTF-8 encoding
\usepackage[T1]{fontenc}    % For proper Italian accents
\usepackage[italian]{babel} % Italian language support
\usepackage{amsmath, amsfonts, amssymb} % Math packages
\usepackage{graphicx}       % For including images
\usepackage{hyperref}       % For hyperlinks
\usepackage{bookmark}       % For improved PDF bookmarks and rerunfilecheck warning
\usepackage{tikz}           % For diagrams
\usepackage{geometry}       % For page layout
\usepackage{fancyhdr}       % For custom headers/footers
\usepackage{enumitem}       % For custom lists
\usepackage{xcolor}         % For colored text
\usepackage{mathtools}      % For additional math tools
\usepackage{amsthm}         % For theorem environments
\usepackage{quiver}         % For commutative diagrams

% Load custom commands
\usepackage{commands} 
% Theorem styles

% Page layout
\geometry{a4paper, margin=1in}
\pagestyle{fancy}
\fancyhf{}
\fancyhead[L]{Note della parte di Topologia Algebrica del corso di Geometria 2 del 2024/25}
\fancyhead[R]{\thepage}

\begin{document}

\title{Appunti di Topologia Algebrica}
\author{Simone Riccio}
\date{\today}

\maketitle

\tableofcontents

\section{Gruppo Fondamentale}
\subsection{Omotopia}
    Una delle motivazioni che porta a definire il gruppo fondamentale è la necessità di distinguere
    due spazi topologici a meno di omeomorfismo.

    \begin{example} \nl
        Si consideri il disco 
        \[  
            D^n := \set{ x \in \R^{n} \mid \norm{x} \leq 1 }
        \]
        Al variare di $n$ naturale i $D^n$ non sono intuitivamente omeomorfi, tuttavia dimostrarlo
        usando solo la topologia generale è difficile.

        È semplice mostrare che $\Disc{1} \not\cong D^n$ per $n \geq 2$, usando l'insieme delle componenti 
        connesse. Infatti, per ogni $x \in D^n$ lo spazio topologico $D^n \setminus \set{x}$ è connesso 
        per ogni $n \geq 2$, mentre $\Disc{1} \setminus \set{x}$, essendo il segmento $[-1,1]$ senza un punto, ha due componenti connesse.
    
        Tale argomentazione non funziona già per provare a distinguere $\Disc{2}$ dai $D^n$ con $n \geq 3$.
        Introduciamo quindi il gruppo fondamentale, che permetterà in futuro di distinguerli tutti.
    \end{example}

    \begin{definition}[Omotopia] \nl
        Date due funzioni continue $f,g: X \to Y$ tra spazi topologici, si dice che $f$ e $g$ sono \textbf{omotope} se esiste una funzione
        \[
            H: I \times X \to Y
        \]
        \textbf{continua} e tale che:
        \begin{itemize}
            \item $H(0,x) = f(x)$ per ogni $x \in X$;
            \item $H(1,x) = g(x)$ per ogni $x \in X$;
            \item $H(s,y) = H(s,x)$ per ogni $s \in I$ e per ogni $x,y \in X$ tali che $f(x) = f(y)$.
        \end{itemize}
        Si dice che $H$ è un'\textbf{omotopia} tra $f$ e $g$ e si scrive
        \[
            f \sim g.
        \]
        Inoltre si pu\`o vedere un'omotopia come una famiglia di funzioni contiune:
        \[
            \set{f_s: X \to Y}_{s \in I}
            \quad \text{con } f_s(x) = H(s,x).
        \]
        Che rappresentano una deformazione continua di $f$ in $g$.
    \end{definition}

    \begin{definition}[Omotopia di cammini a estremi fissi] \nl
        Due cammini $\gamma_0, \gamma_1: I \to X$ si dicono \textbf{omotopi (a estremi fissi)} se esiste una funzione
        \[
            H: I \times I \to X
        \]
        \textbf{continua} e tale che:
        \begin{itemize}
            \item $H(0,t) = \gamma_0(t)$ per ogni $t \in I$;
            \item $H(1,t) = \gamma_1(t)$ per ogni $t \in I$;
            \item $H(s,0) = H(s,1)$ per ogni $s \in I$.
        \end{itemize}
        Si dice che $H$ è un'\textbf{omotopia di cammini a estremi fissi} e si scrive
        \[
            \gamma_0 \sim \gamma_1.
        \] 
        Infatti è facile verificare che l'essere omotopi a estremi fissi induce una relazione di equivalenza
        sull'insieme dei cammini in $X$.
    \end{definition}

    \begin{definition} [Giunzione di cammini] \nl
        Siano $f,g: I \to X$ due cammini in $X$ con $f(1) = g(0)$, allora la \textbf{giunzione} di $f$ e $g$ è il cammino
        \[
            f * g: I \to X: t \mapsto 
            \begin{cases}
                f(2t) & \text{se } 0 \leq t \leq \frac{1}{2}, \\
                g(2t-1) & \text{se } \frac{1}{2} < t \leq 1.
            \end{cases}
        \]
    \end{definition}

    \begin{lemma} [Giunzione di cammini e omotopia] \nl
        Se $f \sim f'$ e $g \sim g'$, allora $f * g \sim f' * g'$.
    \end{lemma}

    \begin{proof}
            Sia $H_f: I \times I \to X$ un'omotopia di $f$ e $f'$ e $H_g: I \times I \to X$ un'omotopia di $g$ e $g'$.

            Definiamo l'omotopia
            \[
                H: I \times I \to X: (s,t) \mapsto 
                \begin{cases}
                    H_f(2s,t) & \text{se } 0 \leq s \leq \frac{1}{2}, \\
                    H_g(2s-1,t) & \text{se } \frac{1}{2} < s \leq 1.
                \end{cases}
            \]
            che risulta continua. Infatti la continuità di $H_f$ e $H_g$ implica la continuità di $H$, essendo le due funzioni
            definite su due intervalli disgiunti. Inoltre si verifica facilmente che $H$ soddisfa le condizioni richieste.
    \end{proof}

    \begin{remark} \nl
        Si noti che la giunzione di cammini non è definita su ogni coppia di cammini, ma solo su quelle che hanno
        il punto finale del primo uguale al punto iniziale del secondo. Tuttavia, se si considerano solo i cammini chiusi \textbf{che partono da uno stesso punto iniziale},
        la giunzione è chiaramente sempre definita.
    \end{remark}

\subsection{Definizione del gruppo fondamentale}
    Da ora in poi gli spazi topologici considerati saranno sempre localmente connessi.

    \begin{theorem} [Poincar\'e] \nl
        Se $X$ uno spazio topologico e $x_0 \in X$ un punto fisso. \nl
        Il prodotto dato dalla giunzione di cammini induce una struttura di gruppo sulle classi di omotopia
        dei cammini chiusi in $X$ aventi punto iniziale $x_0$. \nl
        Tale gruppo è chiamato \textbf{gruppo fondamentale} di $X$ in $x_0$ e si denota con $\pi_1(X,x_0)$. \nl
        In tale gruppo l'elemento neutro è rappresentato dal cammino costante in $x_0$ e l'inverso di un cammino $\gamma$ è il cammino
        \[
            \gamma^{-1}(t) = \gamma(1-t)
        \]
        che è l'inverso rispetto alla giunzione di cammini.
    \end{theorem}

    Per la dimostrazione del teorema di Poincar\'e ci basta dimostrare prima un lemma.

    \begin{lemma} (Riparametrizzazione di un cammino e omotopia) \nl
        Sia $\gamma: I \to X$ un cammino in $X$ e sia $\varphi: I \to I$ una funzione continua tale che $\phi(0) = 0$ e $\phi(1) = 1$.
        Allora $\gamma \circ \varphi: I \to X$ è un cammino in $X$ e $\gamma \sim \gamma \circ \varphi$.
    \end{lemma}

    \begin{proof}
        Basta mostrare che la funzione $\varphi$ \`e omotopa all'identita\`a $id_I$. \nl
        L'omotopia \`e data dalla famiglia di funzioni
        \[
            \varphi_s: I \to I: t \mapsto (1-s) t + s \varphi(t).
        \]
        E poi boh.. buco.
    \end{proof}

    \begin{proof} [Teorema di Poincar\`e] \nl
        \begin{itemize}
            \item (Associatività) \nl
            Siano $\gamma_1, \gamma_2, \gamma_3: I \to X$ tre cammini chiusi in $X$ con punto iniziale $x_0$.
            Si ha che
            \[
                (\gamma_1 * \gamma_2) * \gamma_3 \sim \gamma_1 * (\gamma_2 * \gamma_3).
            \]
            Poich\'e $\gamma_1 * (\gamma_2 * \gamma_3)$ si pu\`o vedere come una Riparametrizzazione
            del cammino $(\gamma_1 * \gamma_2) * \gamma_3$ e quinid usare il lemma.
            \item (Unità) \nl
            L'elemento neutro del gruppo fondamentale è il cammino costante in $x_0$, che si denota con $e: I \to {x_0}$. \nl
            Infatti, per ogni cammino $\gamma: I \to X$ si ha che
            $\gamma * e $ \`e la Riparametrizzazione di $\gamma$ secondo la mappa
            \[
                \varphi: I \to I: t \mapsto \begin{cases}
                    2t & \text{se } 0 \leq t \leq \frac{1}{2}, \\
                    1 & \text{se } \frac{1}{2} < t \leq 1
                \end{cases}.
            \]
            \item (Inverso) \nl
            Sia 
            \[
                \gamma_s: I \to X: t \mapsto
                \begin{cases}
                    \gamma(t) & \text{se } 0 \leq t \leq s, \\
                    \gamma(s) & \text{se } s < t \leq 1.
                \end{cases}
            \]
            La famiglia di cammini $\set{\gamma_s}_{s \in I}$, che non sono lacci, rappresenta un'omotopia tra il cammino costante in $x_0$ e il cammino 
            $\gamma$, tuttavia \textbf{non rappresenta un'omotopia ad estremi fissi} poiché $\gamma_s(1) \neq \gamma(1)$.
            Vale pero\` che $\gamma_s(0) = \gamma(0)$ cio\`e il punto iniziale \`e fisso. \nl
            In modo analogo la famiglia di cammini data da 
            \[
                \gamma_s^{-1}(t) := gamma_s(1-t) 
            \]
            rappresenta un'omotopia tra il cammino costante in $x_0$ e il cammino $\gamma^{-1}$, ma non ad estremi fissi. \nl
            A questo punto si verifica che la famiglia di \textbf{cammini chiusi} $\set{\gamma_s * \gamma_s^{-1}}_{s \in I}$ rappresenta un'omotopia \textbf{ad estremi fissi} 
            tra il cammino costante in $x_0$ e il cammino $\gamma * \gamma^{-1}$. \nl
            Si fa in maniera analoga per mostrare che $\gamma^{-1} * \gamma \sim e_{x_0}$
        \end{itemize}
    \end{proof}

    \begin{example} \nl
        \[
            \pi_1(\R^n, x_0) = \{ e_{x_0} \} \quad \forall x_0 \in \R^n.
        \]

        Siano $\alpha, \beta: I \to \R^n$ due cammini chiusi in $\R^n$ con punto iniziale $x_0$. \nl
        La famiglia di cammini chiusi definita da
        \[
            f_s: I \to \R^n: t \mapsto (1-s) \alpha(t) + s \beta(t)
        \]
        definisce un'omotopia ad estremi fissi tra $\alpha$ e $\beta$. \nl
        Piu\` in generale, l'omotopia definita e\` quella che per ogni punto dei cammini percorre al variare di $s$
        il segmento che unisce i due cammmini in quell'istante $t$, e dunque la stessa argomentazione vale per dimostrare che:
       \[
        \begin{aligned}
            &\forall X \subset \R^n \text{ convesso}, \\
            &\pi_1(X, x_0) = \{ e_{x_0} \} \quad \forall x_0 \in \R^n.
        \end{aligned}
        \]
    \end{example}

    \begin{proposition} [Gruppo fondamentale di un connesso per archi] \nl
        Sia $X$ uno spazio topologico connesso per archi, allora
        \[
            \pi_1(X, x_0) \cong \pi_1(X, x_1) \quad \forall x_0, x_1 \in X.
        \]
        In altre parole, il gruppo fondamentale di uno spazio topologico connesso per archi non dipende dal punto iniziale scelto.
    \end{proposition}

    \begin{proof}
    Sia $f: I \to X$ un cammino tale che $f(0) = x_0$ e $f(1) = x_1$, che esiste poiché $X$ è connesso per archi.
    Tale cammino induce un isomorfismo tra i gruppi fondamentali in $x_0$ e $x_1$:
    \[
        \begin{aligned}
            \pi_1(X, x_0) &\xrightarrow{\sim} \pi_1(X, x_1) \\
            [\gamma] &\mapsto [f * \gamma * f^{-1}]
        \end{aligned}
    \]
    con inversa data da
    \[
        \begin{aligned}
            \pi_1(X, x_1) &\xrightarrow{\sim} \pi_1(X, x_0) \\
            [\gamma] &\mapsto [f^{-1} * \gamma * f].
        \end{aligned}
    \]
    Infatti, si verifica prima di tutto la buona definizione: \nl
    Se $\gamma_1 \sim \gamma_2$ sono due cammini chiusi in $X$, per il lemma della Riparametrizzazione, si ha che
    \[
        f * \gamma_1 * f^{-1} \sim f * \gamma_2 * f^{-1}.
    \]
    Inoltre, si verifica che l'immagine di un cammino chiuso in $x_0$ è un cammino chiuso in $x_1$ e viceversa. \nl
    Si si veririfica che le funzioni appena definite sono effettivamente degli omomorfismi di gruppo
    poiche\` si ha che: 
    \[
        f * \gamma_1 * \gamma_2 * f^{-1} \sim (f * \gamma_1 * f^{-1}) * (f * \gamma_2 * f^{-1})
    \]
    usando l'associatività che anche se non dimostrata vale anche per cammini chiusi. \nl
    Infine, si verifica facilmente che le due mappe sono una l'inversa dell'altra.
\end{proof}

\begin{remark} \nl
    L'isomorfismo tra i due gruppi fondamentali non è canonico, poiché dipende dalla scelta del cammino $f$ tra i due punti $x_0$ e $x_1$.
\end{remark}

\begin{definition} [Spazio semplicemente connesso] \nl
    Uno spazio topologico $X$ si dice \textbf{semplicemente connesso} se è connesso per archi e il suo gruppo fondamentale è banale, cioè
    \[
        \pi_1(X, x_0) = \{ e_{x_0} \} \quad \forall x_0 \in X.
    \]
\end{definition}

\begin{remark} \nl
    Se $X$ è semplicemente connesso e $\alpha, \beta: I \to X$ sono due cammini allora
    \[ \alpha(0) = \beta(0), \quad \alpha(1) = \beta(1) \implies \alpha \sim \beta\]
    Dato che il cammino $\alpha * \beta^{-1}$ è chiuso e il gruppo fondamentale è banale, quindi
    \[
        \alpha * \beta^{-1} \sim e_{x_0} \implies \alpha \sim \beta.
    \]
\end{remark}

\begin{remark} [La funtorialit\`a del gruppo fondamentale] \nl
    Siano $X, Y$ due spazi topologici e $\varphi: X \to Y$ una mappa continua tale che $\varphi(x_0) = y_0$ per due punti fissi $x_0 \in X$ e $y_0 \in Y$. \nl
    Allora $\varphi$ induce un omomorfismo di gruppi
    \[
        \varphi_*: \pi_1(X, x_0) \to \pi_1(Y, y_0)
    \]
    definito da
    \[
        \varphi_*([\gamma]) = [\varphi \circ \gamma]
    \]
    Si verifica facilmente che la mappa \`e ben definita ed \`e un omomorifsmo di gruppi. \nl
    Inoltre, vale che,
    se $\varphi = \id_X$ allora $\varphi_* = id_{\pi_1(X, x_0)}$ e se $(\psi \circ \varphi)_* = \psi_* \circ \varphi_*$. \nl
    Nel linguaggio delle categorie quindi si dice che 
    \[  
        \pi_1: \mathbf{Top} \to \mathbf{Grp}: X \mapsto \pi_1(X, x_0)
    \]
    è un \textbf{funtore} da $\mathbf{Top}$, la categoria degli spazi topologici, a $\mathbf{Grp}$, la categoria dei gruppi.
\end{remark}

\begin{proposition} \nl
    Se $\varphi: X \to Y$ è un omeomorfismo tra spazi topologici, allora
    \[
        \varphi_*: \pi_1(X, x_0) \to \pi_1(Y, y_0)
    \]
    è un isomorfismo di gruppi, dove $x_0 \in X$ e $y_0 = \varphi(x_0) \in Y$.
\end{proposition}

\begin{proof} \nl
    Poich\'e $\varphi$ è un omeomorfismo, essa è continua e ha un'inversa continua $\varphi^{-1}: Y \to X$. \nl
    Cio\`e  $\varphi^{-1} \circ \varphi = \id_X$ e $\varphi \circ \varphi^{-1} = \id_Y$, quindi segue dalla funtorialit\'a
    che
    \[
        \varphi_* \circ \varphi^{-1}_* = id_{\pi_1(X, x_0)} \quad \text{e} \quad \varphi^{-1}_* \circ \varphi_* = id_{\pi_1(Y, y_0)}.
    \]
    Quindi $\varphi_*$ è un isomorfismo di gruppi, poiché ha un'inversa data da $\varphi^{-1}_*$.
\end{proof}

\begin{definition} [Spazi omotopicamente equivalenti] \nl
    Due spazi topologici $X$ e $Y$ si dicono \textbf{omotopicamente equivalenti} se esistono due funzioni continue
    \[
        f: X \to Y \quad \text{e} \quad g: Y \to X
    \]
    tali che:
    \begin{itemize}
        \item $g \circ f$ è omotopa all'identità su $X$;
        \item $f \circ g$ è omotopa all'identità su $Y$.
    \end{itemize}
    Si denota con $X \simeq Y$ se $X$ e $Y$ sono omotopicamente equivalenti.
\end{definition}

\begin{example} \nl
    \begin{enumerate} 
        \item $\R^n$ \`e omotopicamente equivalente ad un punto, cio\`e si dice che $\R^n$ \`e \textbf{contraibile}.
            Infatti sia $\varphi: \R^n \to \set{0} \subset \R^n$ la funzione costante in $0$, che è continua.
            e sia $\psi: \set{0} \to \R^n$ anch'essa continua. \nl
            Si ha che $\varphi \circ \psi = \id_{\set{0}}$, mentre $\psi \circ \varphi$ è omotopa all'identità su $\R^n$ tramite l'omotopia
            definita da
            \[
                H: I \times \R^n \to \R^n: (s,x) \mapsto s x.
            \]
        \item $S^n$ \`e omotopicamente equivalente a $\R^{n+1} \setminus{\set{0}}$ \nl
        Infatti se $i: S^n \into \R^n$ \`e l'inclusione di $S^n$ in $\R^{n+1} \setminus{\set{0}}$ e
        \[  
            \psi: \R^{n+1} \setminus{\set{0}} \to S^n: x \mapsto \frac{x}{\norm{x}}.
        \]
        Si ha che $i \circ \psi = \id_{S^n}$ e $\psi \circ i \sim \id_{\R^{n+1} \setminus{\set{0}}}$ tramite l'omotopia
        \[
            H: I \times \R^{n+1} \setminus{\set{0}} \to \R^{n+1} \setminus{\set{0}}: (s,x) \mapsto (1-s)x + s\frac{x}{\norm{x}}.
        \]
        \item Il Nastro di Möbius \`e omotopicamente equivalente al cerchio $\Sph{1}$. \nl
            Infatti, sia $M$ il Nastro di Möbius e sia $\varphi: M \to \Sph{1}$ la proiezione che manda ogni punto del nastro sul suo bordo.
            Si ha che $\varphi$ è continua e suriettiva. \nl
            Infatti se consideriamo il quadrato $Q = [-1,1] \times [-1,1]$, tale spazio \`e omotopicamente equivalente al segmento
            $[-1,1]$ tramite l'inlcusione del segmento nel quadrato e la proiezione naturale del quadrato sul segmento.
            Identificando i lati opposti del quadrato in modo da ottenere il Nastro di Möbius, si ha che
            la proiezione del quadrato sul segmento induce una mappa continua e suriettiva dal Nastro di Möbius al cerchio, con omotopie 
            che passano al quoziente.
        \end{enumerate}
\end{example}

\begin{theorem} (Spazi omotopicaamente equivalenti hanno gruppo fondamentale isomorfo) \nl
    Siano $X$ e $Y$ spazi topologici \textbf{connessi per archi} omotopicamente equivalenti, allora i loro gruppi fondamentali sono isomorfi:
    \[
        \pi_1(X, x_0) \cong \pi_1(Y, y_0)
    \]
    per ogni coppia di punti fissi $x_0 \in X$ e $y_0 \in Y$.
\end{theorem}

\begin{lemma} \nl
    Siano $\varphi_0, \varphi_1: X \to Y$ due funzioni continue \textbf{omotope} tra spazi topologici e siano $x_0 \in X$. \nl
    Il seguente diagramma commuta:
    % https://q.uiver.app/#q=WzAsMyxbMCwxLCJcXHBpXzEoWF8seF8wKSJdLFsxLDAsIlxccGlmKFNfMSxYLFxcdmFycGhpXzAoeF8wKSkiXSxbMSwyLCJcXHBpXzEoWSxcXHZhcnBoaV8xKHhfMCkpIl0sWzAsMSwiXFx2YXJwaGlfezAqfSJdLFswLDIsIlxcdmFycGhpX3sxKn0iLDJdLFsxLDIsIlxcdGF1X2YiXV0=
    \[
        \begin{tikzcd}
        & {\pi_1\left(Y,\varphi_0\left(x_0\right)\right)} \\
        {\pi_1\left(X_,x_0\right)} \\
        & {\pi_1\left(Y,\varphi_1\left(x_0\right)\right)}
        \arrow["{\tau_f}", from=1-2, to=3-2]
        \arrow["{\varphi_{0*}}", from=2-1, to=1-2]
        \arrow["{\varphi_{1*}}"', from=2-1, to=3-2]
        \end{tikzcd}
    \]
    dove $\tau_f: \pi_1\left(Y, \varphi_0\left(x_0\right)\right) \to \pi_1\left(Y, \varphi_1\left(x_0\right)\right)$ è l'isomorfismo indotto dal cammino
    $f: I \to Y: s \mapsto \varphi_s\left(x_0\right)$ e $\set{\varphi_s \mid s \in I}$ \`e l'omotopia tra $\varphi_0$ e $\varphi_1$.
\end{lemma}

\begin{proof} \nl
    Si consideri la mappa
    \[
        \tau^{-1}_f := \tau_{f^{-1}}: \pi_1\left(Y, \varphi_1\left(x_0\right)\right) \to \pi_1\left(Y, \varphi_0\left(x_0\right)\right): g_Y \mapsto f * g_Y * f^{-1}.
    \]
    Al variare di $s \in I$ si ha che
    \[
        f_s: I \to Y: t \mapsto f\left(st\right)
    \]
   rappresenta un'omotopia tra il cammino $f_0: I \to \set{\varphi_0\left(x_0\right)}$ 
   e il cammino $f$. \nl
   Quindi, se ora si considera $g_X$ un cammino chiuso in $x_0 \in X$, allora la mappa
   \[
        I \to \pi_1\left(Y, \varphi_0\left(x_0\right)\right): s \mapsto f_s * \varphi_{0}\left(g_X\right) * f_s^{-1}
   \]
   induce un'omotopia tra il cammino chiuso $\varphi_{0}\left(g_X\right)$ e il cammino chiuso
    $f\left(\varphi_{1}\left(g_X\right)\right)$, dunque vale  che
    \[
        \varphi_{0*}\left(g_X\right) = \tau_f\left(\varphi_{1*}\left(g_X\right)\right).
    \]
\end{proof}

\begin{proof} [del teorema] \nl
    Siano $\varphi: X \to Y$ e $\psi: Y \to X$ le funzioni continue che definiscono l'equivalenza omotopica tra $X$ e $Y$. \nl
    Grazie al lemma precedenta, dato che vale $\psi \circ \varphi \sim \id$ si ha che il seguente diagramma commuta:
    % https://q.uiver.app/#q=WzAsNCxbMCwwLCJcXHBpXzFcXGxlZnQoWCwgeF8wIFxccmlnaHQpIl0sWzEsMCwiXFxwaV8xXFxsZWZ0KFksIFxcdmFycGhpXFxsZWZ0KHhfMFxccmlnaHQpIFxccmlnaHQpIl0sWzIsMCwiXFxwaV8xXFxsZWZ0KFksIChcXHBzaSBcXGNpcmMgXFx2YXJwaGkpXFxsZWZ0KHhfMFxccmlnaHQpIFxccmlnaHQpIl0sWzIsMSwiXFxwaV8xXFxsZWZ0KFgsIHhfMCBcXHJpZ2h0KSJdLFswLDMsIlxcaWQiLDJdLFswLDEsIlxcdmFycGhpXyoiXSxbMSwyLCJcXHBzaV8qIl0sWzIsMywiXFx0YXVfZiJdXQ==
    \[
    \begin{tikzcd}
        {\pi_1\left(X, x_0 \right)} & {\pi_1\left(Y, \varphi\left(x_0\right) \right)} & {\pi_1\left(X, (\psi \circ \varphi)\left(x_0\right) \right)} \\
        && {\pi_1\left(X, x_0 \right)}
        \arrow["{\varphi_*}", from=1-1, to=1-2]
        \arrow["\id"', from=1-1, to=2-3]
        \arrow["{\psi_*}", from=1-2, to=1-3]
        \arrow["{\tau_f}" right, "{\cong}" left, from=1-3, to=2-3]
    \end{tikzcd}
    \]
    Cio\`e vale che $ \tau_f \circ \psi_* \circ \varphi_* = \id$, quindi $\psi_* \circ \varphi_* = \tau_f^{-1}$, ma se la composizione di due mappe
    \`e bigettiva allora la prima $\varphi_*$ \`e iniettiva e la seconda $\psi_*$ \`e suriettiva, ragionando in maniera analoga
    per il verso opposto si ha che $\varphi_* \circ \psi_* = \tau_f$ e quindi $\psi_*$ \`e iniettiva e $\varphi_*$ \`e surgettiva. \nl
    Si conclude quindi che $\varphi_*$ e $\psi_*$ sono isomorfismi di gruppi.
         
\end{proof}

\subsection{Primi gruppi fondamentali}

Da questo momento in poi, se $X$ \`e uno spazio topologico connesso per archi, si denota con $\pi_1(X)$ il gruppo fondamentale di uno spazio topologico $X$ in un punto fissato.

\begin{theorem} [Gruppo fondamentale del cerchio] \nl
    \[
        \pi_1\left(\Sph{1} \right) \cong \Z.
    \]
    Ed il cammino chiuso $t \mapsto e^{2\pi i t}$ rappresenta il generatore del gruppo fondamentale $\pi_1\left(\Sph{1}, 1\right)$.
\end{theorem}

\begin{definition} [Mappa esponenziale] \nl
    Si definisce la mappa
    \[
        \rho: \R \to \Sph{1}: t \mapsto e^{2\pi i t}
    \]
\end{definition}

\begin{lemma} [Sollevamento di un cammino di $\Sph{1}$ in $\R$] \nl
    \begin{enumerate}
        \item Per ogni cammino chiuso $f: I \to \Sph{1}$ con $f(0) = f(1)$, 
        \textbf{esiste ed unico} un cammino(in generale non chiuso) $\tilde{f}: I \to \R$ detto \textbf{sollevamento di $f$ in $\R$} tale che
        \[
            \tilde{f}(0) = 0 \quad \text{e} \quad \rho \circ \tilde{f} = f.
        \]Il terminale mi dice di aver committato, ma su github la repository non sembra essere commi
        Ovvero il seguente diagramma commuta:
        \[\begin{tikzcd}
            I & \R \\
            & {\Sph{1}}
            \arrow["{\tilde{f}}", from=1-1, to=1-2]
            \arrow["f"', from=1-1, to=2-2]
            \arrow["\rho", from=1-2, to=2-2]
        \end{tikzcd}\]
        \item Inoltre se $f_0, f_1$ sono due cammini chiusi omotopi allora
        \[
            \tilde{f}_0(1) = \tilde{f}_1(1) \in \Z
        \] 
    \end{enumerate}
\end{lemma}

\begin{proof} (Lemma $\implies$ Teorema) \nl
    Dal lemma segue che la mappa:
    \[
        \Phi: \pi_1\left(\Sph{1}, 1\right) \to \Z: [f] \mapsto \tilde{f}(1)
    \]Il terminale mi dice di aver committato, ma su github la repository non sembra essere commi
    \`e ben definita, ed inoltre induce un omomorfismo di gruppi, poich\'e
    \[
        \Phi\left([\gamma_1 * \gamma_2]\right) = \tilde{\gamma}_1(1) + \tilde{\gamma}_2(1) = \Phi\left([\gamma_1]\right) + \Phi\left([\gamma_2]\right).
    \].
    Si dimostra ora la surgettivit\`a di $\Phi$, infatti dato il cammino chiuso $f_1: I \to \Sph{1}: t \mapsto e^{2\pi i t}$, si ha che
    \[
        \Phi\left([f_1^n]\right) = n \tilde{f}_1(1) = n.
    \]
    Infine, si verifica che il nucleo di $\Phi$ \`e l'insieme dei cammini chiusi omotopi al cammino costante in $1$, dato che
    se $f: I \to \Sph{1}$ \`e un cammino chiuso tale che $\tilde{f}(1) = 0$, si ha che $\tilde{f}$ \`e un cammino chiuso in $\R$ che parte da $0$ e torna a $0$, quindi
    poich\'e $\R$ \`e semplicemente connesso, esiste un'omotopia $H: I \times I \to \R$ da $\tilde{f}$ al camminio costante in $0$.
    Ma a questo punto si ha che $\rho \circ H$ è un'omotopia da $f$ al cammino costante in $1$, quindi $f$ è omotopo al cammino costante in $1$.
\end{proof}

\begin{proof} (del teorema) \nl
    OK LA DIMOSTRAZIONE DI QUESTO FATTO FATTA DA TAMAS \`E RIDICOLA, MEGLIO FARE QUELLA PIU\' GENERALE DI FRIGERIO QUANDO SAR\`A POI
    \begin{enumerate}
        \item si consideri il ricoprimento di aperti di $\Sph{1}$ dato da due aperti $U_0, U_1$, archi che si intersecano in due componenti 
        connesse per archi di $\Sph{1}$, una che contiene $1$ e l'altra che contiene $-1$. 
        \nl
        DISEGNO DA FARE
        \nl
        Considero le componenti connesse per archi di $\rho^{-1}(U_1)$, che formano un ricorpimento di aperti
        per $\rho^{-1}(U_1)$. La mappa $\rho$ induce un omeomorfismo 
        $\rho\big|_{\rho^{-1}(U_1)}: \rho^{-1}(U_1) \to U_1$(si vedr\`a che \`e un rivestimento di $U_1$)).
        \nl
        Inoltre, le componenti connesse per archi di $f^{-1}(U_0), f^{-1}(U_1)$ formano un ricoprimento di aperti per $I$.
        \nl
        L'intervallo $I$ \`e uno spazio metrico compatto, e dunque ammette un numero di Lebesgue. \nl
        Siano quindi $t_0 = 0 < t_1 < t_2 < \ldots < t_n = 1$ i punti di $I$ tali che
        ciascun intervallo della $\left[t_i, t_{i+1}\right]$ \`e interamente contenuto in uno
        ed uno solo tra  $f^{-1}(U_0)$ e $f^{-1}(U_1)$ e inoltre $t_i \in f^{-1}(U_0) \cap f^{-1}(U_1) \quad \forall i$.
        \nl
    \end{enumerate}
\end{proof}

\begin{corollary} \nl
    \[  
        \pi_1(\R^n \setminus \set{0}) \cong \pi_1(D^n \setminus \set{0}) \cong \Z
    \]
    In particolare $\Disc{2} \setminus \set{0}$ non \`e omotopicamente equivalente $\Disc{2}$(e quindi nemmeno omeomorfo).
\end{corollary}

\begin{definition} [Retrazione] \nl
    Sia $Y \subset X$ un sottospazio topologico. Si dice che una mappa continua $r: X \to Y$ \`e una retraazione se vale
    \[
        r \circ i = \id_Y
    \]
    dove $i: Y \into X$ \`e l'inclusione di $Y$ in $X$. \nl 
    In altre parole, $r$ \`e una retrazione se \`e continua e manda ogni punto di $Y$ su se stesso. \nl
    Si dice che $Y$ \`e \textbf{retratto} in $X$ se esiste una retrazione da $X$ a $Y$.
\end{definition}

\begin{example}
    \begin{enumerate}
        \item  In ogni spazio topologico $X$ ogni punto ${x_0} \subset X$ \`e un retratto di $X$.
        \item  Il segmento $I = [-1, 1]$ \`e un retratto di $\bar{\Disc{2}} = \closedball{0}{1}$, infatti la mappa
            \[
                r: \bar{\Disc{2}} \to I: (x,y) \mapsto x
            \]
            \`e una retrazione, poich\'e $r$ manda ogni punto del segmento su se stesso.

    \end{enumerate}
\end{example}

\begin{lemma} [Retrazione e gruppo fondamentale] \nl
    Sia $Y \subset X$ un retratto di $X$ e sia $x_0 \in Y$. Allora la mappa indotta dall'inclusione naturale
    \[
        i_*: \pi_1(Y, x_0) \to \pi_1(X, x_0)
    \]
    \`e un omomorfismo di gruppi \textbf{iniettivo}
\end{lemma}

\begin{proof} \nl
    \[\begin{tikzcd}
	{\pi_1\left(Y, x_0\right)} & {\pi_1\left(X, x_0\right)} & {\pi_1\left(Y, x_0\right)}
	\arrow["{i_*}", from=1-1, to=1-2]
	\arrow["\id"', curve={height=18pt}, from=1-1, to=1-3]
	\arrow["{r_*}", from=1-2, to=1-3]
    \end{tikzcd}\]
    Dunque $r_* \circ i_* = \id_{\pi_1(Y, x_0)}$, quindi $i_*$ \`e iniettiva perch\'e ha inversa sinistra. \nl
\end{proof}

\begin{corollary} \nl
    $\Sph{1}$ non \`e un retratto di $\bar{\Disc{2}}$.
\end{corollary}

\begin{proof}
    \[\begin{tikzcd}
	\Z & {\pi_1\left(\Sph{1}, 1\right)} & {\pi_1\left(\bar{\Disc{1}}, 1\right)} & {\set{1}}
	\arrow["\cong"{description}, draw=none, from=1-1, to=1-2]
	\arrow["{i_*}", from=1-2, to=1-3]
	\arrow["\cong"{description}, draw=none, from=1-3, to=1-4]
\end{tikzcd}\]
    e quindi $i_*$ non \`e iniettiva, perch\'e \`e forzata ad essere banale. 
\end{proof}

\begin{theorem} [Brouwer] \nl
    Ogni applicazione continua $f: \Disc{2} \to \Disc{2}$ ammette un punto fisso.
\end{theorem}

\begin{proof}
    La dimostrazione \`e per assurdo. \nl
    Si supponga che per ogni punto $x \in \Disc{2}$ si ha che $f(x) \neq x$. \nl
    Consideriamo la mappa continua
    \[
        r: \Disc{2} \to \Sph{1}: x \mapsto \frac{f(x) - x}{\norm{f(x) - x}}.
    \]
    Questa mappa associa ad ogni punto $x \in \Disc{2}$ un punto sulla circonferenza unitaria $\Sph{1}$, che rappresenta la direzione del vettore che punta da $x$ a $f(x)$. \nl
    Una tale mappa sarebbe una retrazione del disco unitario $\Disc{2}$ su $\Sph{1}$, poiché ogni punto di $\Sph{1}$ sarebbe raggiunto da un punto di $\Disc{2}$ che non si mappa su se stesso. 
    $\contradiction$
\end{proof}

\begin{theorem} [Fondamentale dell'algebra] \nl
    Ogni $f(x) \in \C[x]$ polinomio di grado $n \geq 1$ ammette almeno una radice complessa.
\end{theorem}

\begin{proof} \nl
    Si supponga $f(z) \neq 0 \quad \forall z \in \C$, allora $\forall r > 0$ la mappa
    \[
        f_r(t) := \frac{f\left(r \cos(2\pi t) + ir\sin(2 \pi t)\right)}{\abs{f\left(r \cos(2\pi t) + ir\sin(2 \pi t)\right)}}
    \]
    definisce un cammino chiuso in $I \to \Sph{1}$ che parte da $1$. \nl
    La famiglia di cammini chiusi $\set{f_r | r \in I}$ rappresenta un'omotopia tra il cammino costante $f_0: I \to \set{1}$ e il cammino chiuso
    $f_1$. \nl
    Componendo inoltre con la mappa $I \to [0,r]: s \to sr$ otteniamo un'omotopia ad estremi fissi tra il cammino costante e il cammino chiuso $f_r$. \nl
    Quindi $\left[f_r^0 \right] = 0 \in \Z$. Vogliamo ora dimostrare che $\left[f_r^1 \right] \neq 0 \in \Z$  per ogni $r > 0$ e raggiungere una contraddizione. \nl
    Sia ora il polinomio
    \[
        f(z) = z^n + a_{n-1}z^{n-1} + \ldots + a_0
    \]
    e per $s \in I$ si consideri
    \[
        f_r^s(z) = z^n + s\left( a_{n-1}z^{n-1} + \ldots + a_0 \right).
    \]
    Se $r > \max\set{1, \sum \abs{a_i}}$ e $\abs{z} = r$, allora
    \[
        \abs{z^n} = r^n > s \left( \sum \abs{a_i} \right) \abs{z^{n-1}} \geq \abs{s\left(a_{n-1}z^{n-1} + \dots + a_0\right)}
    \]
    e quindi poich\'e vi \`e il maggiore stretto se $\abs{z} = r$, $f_r^s(z) \neq 0$ per ogni $s \in I$. \nl
    In particolare vale 
    \[
        f_r^s(r \cos{2\pi t} + ir\sin(2 \pi t) \neq 0 
    \]
    E quindi \`e ben definita la famiglia di cammini chiusi
    \[
        f_r^s: I \to \Sph{1}: t \mapsto \frac{f_r^s\left(r \cos{2\pi t} + ir\sin(2 \pi t)\right)}{\abs{f_r^s\left(r \cos{2\pi t} + ir\sin(2 \pi t)\right)}}.
    \]
    che da un'omotopia tra $f_r^0$ e $f_r^1$. \nl
    $f^0 = z^n$ e quindi $f_r^0(t) = \cos(2 n \pi t) + i\sin(2 n \pi t)$
    ma si avrebbe quindi che la classe di omotopia di $f_r^0$ \`e $n \in \Z$, ma ci\`o contraddice il fatto che
    $\left[f_r^0 \right] = 0 \in \Z$.
\end{proof}

\subsection{Teorema di Seifert-Van Kampen e richiami di teoria dei gruppi}

\begin{theorem} [debole di Seifert-van Kampen] \nl
    Siano $X = X_1 \cup X_2$ dove $X_1, X_2 \subset X$ sono aperti. Siano $i_1: X_1 \into X$ e $i_2: X_2 \into X$ le inclusioni naturali. \nl
    Si suppongano $X, X_1, X_2, X_1 \cap X_2$ connessi per archi allora:
    \[
        \pi_1(X, x_0) \text{\`e generato da } i_{1*}\left(\pi_1\left(X_1, x_0\right)\right) \text{ e } i_{2*}\left(\pi_1\left(X_2, x_0\right)\right) \text{dove } x_0 \in X_1 \cap X_2.
    \]
\end{theorem}

\begin{proof}

\end{proof}

\begin{corollary} \nl
    $\Sph{n}$ \`e semplicemente connesso per ogni $n \geq 1$.'
\end{corollary}

\begin{corollary} \nl
    $\R^n \setminus \set{0}$ \`e omotopicamente equivalente a $\Sph{n-1}$ per ogni $n \geq 2$. \nl
    Quindi $\pi_1(\R^2 \setminus \set{0}) \cong \Z$ e $\pi_1(\R^n \setminus \set{0}) \cong \set{1}$ per ogni $n \geq 2$
\end{corollary}

\begin{corollary} \nl
    $\R^2$ non \`e omeomorfo a $\R^n$ per ogni $n \geq 3$.
\end{corollary}

\begin{theorem} (di Seifert-van Kampen) \nl
    Siano $X = X_1 \cup X_2$ dove $X_1, X_2 \subset X$ sono aperti. \nl
    Siano $j_1: X_1 \cap X_2 \into X_1, j_2, X_1 \cap X_2 \into X_2, i_1: X_1 \into X, i_2: X_2 \into X$ le inclusioni naturali. \nl
    Si suppongano $X, X_1, X_2, X_1 \cap X_2$ connessi per archi allora

    $\forall G$ gruppo, e mappe $\varphi_1: \pi_1(X_1 \cap X_2, x_0) \to G$ e $\varphi_2: \pi_1(X_2, x_0) \to G$ esiste un unica mappa
    \[
        \varphi: \pi_1(X, x_0) \to G
    \]
    tale che il seguente diagramma commuta:
    % https://q.uiver.app/#q=WzAsNSxbMSwwLCJcXHBpXzFcXGxlZnQoWF8xLCB4XzBcXHJpZ2h0KSJdLFsxLDIsIlxccGlfMVxcbGVmdChYXzIsIHhfMFxccmlnaHQpIl0sWzAsMSwiXFxwaV8xXFxsZWZ0KFhfMSBcXGNhcCBYXzIsIHhfMFxccmlnaHQpIl0sWzIsMSwiXFxwaV8xXFxsZWZ0KFgsIHhfMFxccmlnaHQpIl0sWzMsMSwiRyJdLFswLDQsIlxcdmFycGhpXzEiXSxbMCwzLCJpX3sxKn0iLDJdLFsxLDMsImlfezIqfSJdLFszLDQsIlxcZXhpc3RzIVxcdmFycGhpIiwwLHsiY29sb3VyIjpbMCw2MCw2MF19LFswLDYwLDYwLDFdXSxbMiwwLCJqX3sxKn0iXSxbMiwxLCJqX3syKn0iLDFdLFsxLDQsIlxcdmFycGhpXzIiXV0=
    \[\begin{tikzcd}
	& {\pi_1\left(X_1, x_0\right)} \\
	{\pi_1\left(X_1 \cap X_2, x_0\right)} && {\pi_1\left(X, x_0\right)} & G \\
	& {\pi_1\left(X_2, x_0\right)}
	\arrow["{i_{1*}}"', from=1-2, to=2-3]
	\arrow["{\varphi_1}", from=1-2, to=2-4]
	\arrow["{j_{1*}}", from=2-1, to=1-2]
	\arrow["{j_{2*}}"{description}, from=2-1, to=3-2]
	\arrow["{\exists!\varphi}", color={rgb,255:red,214;green,92;blue,92}, from=2-3, to=2-4]
	\arrow["{i_{2*}}", from=3-2, to=2-3]
	\arrow["{\varphi_2}", from=3-2, to=2-4]
    \end{tikzcd}\]
\end{theorem}

Il teorema di Van-Kampen necessit\`a di un richiamo di teoria dei gruppi, che non \`e stato ancora fatto. \nl

\begin{definition} [Gruppo libero generato da un insieme] \nl
    Dato un insieme $X$ si indica $F(X)$ il \textbf{gruppo libero generato da $X$} il dato di
    un gruppo $F(X)$ ed una mappa iniettiva $s: X \into F(X)$ tale che la seguente propriet
    `a universale sia soddisfatta: \nl 
    Per ogni gruppo $G$ e per ogni mappa iniettiva $f: X \into G$ esiste un unico omomorfismo di gruppi
    \[
        \bar{f}: F(X) \to G
    \]
    tale che il seguente diagramma commuta:

    \[\begin{tikzcd}
	{F(X)} & G & {} \\
	X
	\arrow["{\exists!\bar{f}}", from=1-1, to=1-2]
	\arrow["s", from=2-1, to=1-1]
	\arrow["f"', from=2-1, to=1-2]
    \end{tikzcd}\]

\end{definition}

\begin{proposition} [Unicit\`a del gruppo libero] \nl
    Dalla definizione di gruppo libero via propriet\`a universale ne segue l'unicit\`a a meno di isomorfismo di gruppi.
\end{proposition}

\begin{proof}
    Facile provaci un attimo
\end{proof}

\begin{proposition} [Costruzione del gruppo libero generato da un insieme] \nl
    Si costruisce ora $F(X)$ nel seguente modo: \nl
    Sull'insieme
    \[
        F(X) = \set{w \in X^* \mid w \text{ parola su } X} / \sim
    \]
    dove una parola $w \in X$ \`e una sequenza un prodotto formale tra simboli della fomra
    \[
        w = x_{i_1}^{\epsilon_1} x_{i_2}^{\epsilon_2} \ldots x_{i_n}^{\epsilon_n}
    \]
    con $x_i \in X$ e $\epsilon_i \in \set{1, -1}$, e la relazione di equivalenza $\sim$ identifica due parole
    se e solo se sono uguali a meno di semplificare i fattori di forma $x_i^{\epsilon_i} x_i^{-\epsilon_i}$. \nl
    L'operazione di gruppo su $F(X)$ \`e data dalla concatenazione formale di parole. \nl

    Tale costruzione verifica la propriet\`a universale del gruppo libero generato da $X$.
\end{proposition}

\begin{proof} \nl
    Per ogni mappa $f: X \into G$ in un gruppo $G$ si definisce la mappa
    \[
        \bar{f}: F(X) \to G: x_{i_1}^{\epsilon_1} x_{i_2}^{\epsilon_2} \ldots x_{i_n}^{\epsilon_n} \mapsto f(x_1)^{\epsilon_1} f(x_2)^{\epsilon_2} \ldots f(x_n)^{\epsilon_n}
    \]
\end{proof}

\begin{lemma} \nl
    Ogni gruppo $G$ \`e il quoziente di un gruppo libero.
\end{lemma}

\begin{proof} \nl
    Se $\set{g_i \mid i \in I}$ si considera $X = \set{x_i \mid i \in I}$ e la mappa
    \[
        \Phi: F(X) \to G: x_i \to g_i \quad \text{ assegnamento per generatori}
    \]
    se i $g_i$ sono un insieme di generatori per $G$ allora $\Phi$ \`e surgettiva e si conclude
    per il primo teorema di omomorfismo tra gruppi.
\end{proof}

\begin{definition} [Presentazione di un gruppo] \nl
    Data $\Phi$ come sopra, sia $N := \ker{\Phi}$ si pu\`o considerare un \textbf{sisterma di generatori per N come sottogruppo normale}
    $\set{p_j \mid j \in J}$, la \textbf{presentazione tramite generatori e relazioni di G} \`e la seguente:
    \[
        G := \left< g_i, \ i \in I \mid p_j, \ j \in J \right>
    \]
\end{definition}

\begin{definition} [Prodotto libero di gruppi] \nl
    Siano $G_1, G_2$ due gruppi, si definisce il \textbf{prodotto libero di gruppi} $G_1 * G_2$
    il dato di un gruppo $G_1*G_2$ e mappe $\gamma_1: G_1 \to _1*G_2$, $\gamma_2: G_2 \to _1*G_2$
    che soddisfano la seguente propriet\`a universale: \nl
    Per ogni altro gruppo $G$ e mappe $\phi_1: G_1 \to G$ e $\phi_2: G_2 \to G$
    esiste un'unica mappa $\phi: G_1 * G_2 \to G$ tale che il seguente diagramma commuti:
     \[
    \begin{tikzcd}
    {G_1} \\
    & {G_1*G_2} & {\textcolor{blue}{G}} \\
    {G_2}
    \arrow["{\gamma_1}", from=1-1, to=2-2]
    \arrow["{\varphi_1}", color={rgb,255:red,92;green,92;blue,214}, curve={height=-12pt}, from=1-1, to=2-3]
    \arrow["{\exists!\varphi}", color={rgb,255:red,214;green,92;blue,92}, dashed, from=2-2, to=2-3]
    \arrow["{\gamma_2}"', from=3-1, to=2-2]
    \arrow["{\varphi_2}", color={rgb,255:red,92;green,92;blue,214}, curve={height=12pt}, from=3-1, to=2-3]
    \end{tikzcd}
    \]
    In teoria delle categorie tale costruzione \`e detta \textbf{coprodotto}.
\end{definition}

\begin{proposition} [Unicit\`a del prodotto libero di gruppi] \nl
    Dalla definizione via propriet\`a universale segue che il prodotto libero \`e unico a meno di isomorfismo di gruppi.
\end{proposition}

\begin{proof}
    Da fare, facile
\end{proof}

\begin{proposition} [Costruzione del prodotto libero tra gruppi]
    Siano 
    \[
        G_1 = \left< g_i^1, \ i \in I_1 \mid p_j^1, \ j \in J_1 \right> \quad G_2 = \left< g_i^2, \ i \in I_2 \mid p_j^2, \ j \in J_2 \right>
    \]
    le due presentazioni dei gruppi, allora la presentazione del prodotto libero \`e data da:
    \[
        G_1*G_2 = \left< \set{g_i^1} \cup \set{g_i^2} \bigm| \set{p_j^1} \cup \set{p_j^2} \right>
    \]

\end{proposition}

\begin{remark} \nl
    \begin{enumerate}
        \item Il gruppo libero \`e generato dalle immagini dei generatori dei gruppi fattori.
        \item Gli elementi di $G_1*G_2$ sono parole in $G_1 \cup G_2$
        \item Se $X_1, X_2$ sono due insiemi allora 
            \[
                F(X_1 \cup X_2) = F(X_1) * F(X_2)
            \]
    \end{enumerate}
\end{remark}

\begin{definition} [Prodotto amalgamato di gruppi] \nl
    Siano $G_1, G_2$ due gruppi e $H$ un terzo gruppo, si definisce il \textbf{prodotto amalgmato di gruppi su H} $G_1 *_H G_2$
    il dato di un gruppo $G_1 *_H G_2$ e mappe $\beta_1: H \to G_1$, $\beta_2: H \to G_2, \gamma_1: G_1 \to _1*G_2$, $\gamma_2: G_2 \to _1*G_2$
    che soddisfano la seguente propriet\`a universale: \nl
    Per ogni altro gruppo $G$ e mappe $\phi_1: G_1 \to G$ e $\phi_2: G_2 \to G$
    esiste un'unica mappa $\phi: G_1 * G_2 \to G$ tale che il seguente diagramma commuti:
    % https://q.uiver.app/#q=WzAsNSxbMSwyLCJHXzIiXSxbMSwwLCJHXzEiXSxbMiwxLCJHXzEqR18yIl0sWzMsMSwiRyJdLFswLDEsIkgiXSxbMSwyLCJcXGdhbW1hXzEiXSxbMCwyLCJcXGdhbW1hXzIiLDJdLFsxLDMsIlxcdmFycGhpXzEiLDAseyJjdXJ2ZSI6LTIsImNvbG91ciI6WzI0MCw2MCw2MF19LFsyNDAsNjAsNjAsMV1dLFswLDMsIlxcdmFycGhpXzIiLDAseyJjdXJ2ZSI6MiwiY29sb3VyIjpbMjQwLDYwLDYwXX0sWzI0MCw2MCw2MCwxXV0sWzIsMywiXFxleGlzdHMhXFx2YXJwaGkiLDAseyJjb2xvdXIiOlswLDYwLDYwXSwic3R5bGUiOnsiYm9keSI6eyJuYW1lIjoiZGFzaGVkIn19fSxbMCw2MCw2MCwxXV0sWzQsMSwiXFxiZXRhXzEiXSxbNCwwLCJcXGJldGFfMiJdXQ==
    \[\begin{tikzcd}
        & {G_1} \\
        H && {G_1*G_2} & G \\
        & {G_2}
        \arrow["{\gamma_1}", from=1-2, to=2-3]
        \arrow["{\varphi_1}", color={rgb,255:red,92;green,92;blue,214}, curve={height=-12pt}, from=1-2, to=2-4]
        \arrow["{\beta_1}", from=2-1, to=1-2]
        \arrow["{\beta_2}", from=2-1, to=3-2]
        \arrow["{\exists!\varphi}", color={rgb,255:red,214;green,92;blue,92}, dashed, from=2-3, to=2-4]
        \arrow["{\gamma_2}"', from=3-2, to=2-3]
        \arrow["{\varphi_2}", color={rgb,255:red,92;green,92;blue,214}, curve={height=12pt}, from=3-2, to=2-4]
    \end{tikzcd}\]
\end{definition}

\begin{proposition} [Costruzione del prodotto amalgamato se $H$ \`e un gruppo libero] \nl
    Siano
    \[
        G_1 = \left< g_i^1, \ i \in I_1 \mid p_j^1, \ j \in J_1 \right> \quad G_2 = \left< g_i^2, \ i \in I_2 \mid p_j^2, \ j \in J_2 \right>
    \]
    \[
        H =  \left< g_i^1, \ i \in I_1 \right> \text{ che non ha relazioni perch\'e libero}
    \]
    allora
    \[
        G_1 *_H G_2 = \left< \set{g_i^1} \cup \set{g_i^2} \bigm| \set{p_j^1} \cup \set{p_j^2} \cup \set{\beta_1(h_i)\beta_2(h_i)^{-1}} \right>
    \]
\end{proposition}

\section{Rivestimenti}

\end{document}
